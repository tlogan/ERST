\documentclass[table,dvipsnames,acmsmall]{acmart}
% \documentclass[letterpaper]{llncs}
% \usepackage[letterpaper, margin=1.5in]{geometry}


\usepackage{multicol}
\usepackage{mathpartir}
\usepackage{hyperref}
\usepackage{mathtools}
\usepackage{amsmath}
\usepackage{nccmath}
\usepackage{stmaryrd}
\usepackage{listings}
\usepackage[scaled]{beramono}
\usepackage[T1]{fontenc}
\usepackage[table,dvipsnames]{xcolor}
\usepackage{array}
\usepackage{wasysym}

\usepackage{pifont}% http://ctan.org/pkg/pifont

\usepackage{graphicx}
\graphicspath{ {./images/} }

\usepackage{url}

\newcounter{pdc}
\newcounter{sdc}

\makeatletter % allow us to mention @-commands
\def\arcr{\@arraycr}
\makeatother

\lstset{
    % identifierstyle=\color{violet},
    % textcolor=blue,
    % keywordstyle=\color{blue},
    % keywordstyle=\text,
    basicstyle=\ttfamily\small,
    % mathescape=true,
    % showspaces=false,
    % morekeywords={def, fix, in}
}

\theoremstyle{definition}
\newtheorem{theorem}{Theorem}[section]
\newtheorem{requirement}{Requirement}[section]
\newtheorem{definition}{Definition}[section]
\newtheorem{property}{Property}[section]
% \newtheorem{proof}{Proof}[section]

\newtheorem{experiment}{Experiment}[section]


\title{Fully Extrinsic Type Inference (DRAFT 2025-06-20)}
\author{Thomas Logan}



\begin{document}



\newcommand{\stoprule}{
  \addlinespace[2pt]
  \toprule
  \addlinespace[5pt]
}

\newcommand{\sbottomrule}{
  \addlinespace[2pt]
  \bottomrule
  \addlinespace[2pt]
}

\newcommand{\smidrule}[1]{
  \addlinespace[2pt]
  \midrule
  \addlinespace[2pt]
}

\newcommand{\scmidrule}[1]{
  \addlinespace[2pt]
  \cmidrule{#1}
  \addlinespace[2pt]
}


\newcommand{\sline}{
  \arrayrulecolor{gray!50}
  \specialrule{0.4pt}{2pt}{2pt}
  \arrayrulecolor{black}
}

\newcommand{\pdc}{\stepcounter{pdc}\arabic{pdc}}
\newcommand{\sdc}{\stepcounter{sdc}\arabic{sdc}}

\newcommand{\Par}[1]{\paragraph{\textbf{#1}}}
\newcommand{\J}[1]{\texttt{\fontfamily{pcr}\selectfont #1}}
\newcommand{\lab}[1]{\text{\color{Gray}\ [#1]}}
\newcommand{\entails}{\vdash}
\newcommand{\satisfies}{\vDash}
\newcommand{\given}{\dashv}
\newcommand{\with}{\ \diamond\ }
\newcommand{\notfree}{\ \#\ }
\newcommand{\consis}{\ \star}
\newcommand{\safe}{\ \checkmark}
\newcommand{\relational}{\ \Re}


% \newcommand{\cmark}{\ding{51}}%
% \newcommand{\xmark}{\ding{55}}%
\newcommand{\pass}{\text{\scriptsize \color{Green} \ding{51}}}
\newcommand{\assisted}{\text{\scriptsize \color{Purple} \ding{45}}}
\newcommand{\fail}{\text{\scriptsize \color{red} \ding{55}}}

\newcommand{\unique}{\ \dagger}

\newcommand{\factorsinto}{\Vdash}


\newcommand{\allsafe}{\ \Re\checkmark}

\newcommand{\qua}{\ .\ }



\newcommand{\ignore}{\varnothing}
\newcommand{\dTheta}{\dot{\Theta}}
\newcommand{\closed}{\bullet}
\newcommand{\local}{\blacktriangle}
\newcommand{\open}{\circ}



\newcommand{\up}{\ \land\ }
\newcommand{\down}{\ \lor\ }

% \newcommand{\tl}{\textasciitilde{}}

\newcommand{\multi}[1]{\widebar{\ #1\ }}
\newcommand{\hastype}{:}
\newcommand{\pattype}{\ \lozenge\ }
\newcommand{\liftfun}{:}

\newcommand{\sz}[1]{\text{\small #1}}


\newcommand{\nopad}{
  \setlength{\abovedisplayskip}{4pt}
  \setlength{\belowdisplayskip}{4pt}
}

\newcommand{\examp}[1]{{\nopad \footnotesize 
\[
\begin{array}[t]{@{}l}
#1
\end{array}
\]
}}


\newcommand{\cb}[1]{
$
\begin{array}[t]{@{}l}#1\end{array}
$
}




\newcommand{\ms}[1]{{\footnotesize $(#1)$}}

\newcommand{\code}[1]{\ms{\J{#1}}}


\newcommand{\R}[1]{\color{teal}\ \ [\emph{#1}]}
 
\newcommand{\subtypes}{<:}
\newcommand{\supertypes}{:>}
\newcommand{\I}{\hspace{4mm}}
\newcommand{\Z}{.\hspace{4mm}}
\newcommand{\Alpha}{\mathrm{A}}
\newcommand{\Tau}{\mathrm{T}}
\newcommand{\B}[1]{\textbf{#1}}
\newcommand{\F}[1]{\text{#1}}
\newcommand{\bigand}{\bigwedge\nolimits}
\newcommand{\bigor}{\bigvee\nolimits}
\newcommand{\C}[1]{\color{teal} \rhd\ \emph{#1}}
\newcommand{\com}[1]{\I \emph{#1}}
\newcommand{\D}[1]{\small \textsc{#1}}
% \newcommand{\fig}[1]{Fig. {\color{red} \fig{#1}}}
\newcommand{\FIG}[1]{Fig. {\color{red} \ref{#1}}}
\newcommand{\TODO}[1]{\noindent \B{\color{red} TODO: #1}}

\newcommand{\is}{\ ::=\ }
\newcommand{\sep}{\ \ |\ \ }
\newcommand{\nonterm}[1]{#1\ }
\newcommand{\contin}{|\ \ \ \ \ \ \ }
\newcommand{\wrt}{\B{wrt }}
% \newcommand{\let}{\B{def }}
% \newcommand{\for}{\B{for }}



\newcommand{\pc}{\hfill \break \noindent \ $\diamond$ \ }


\newcommand{\N}{\ $\cdot$\ \ }

\newcommand{\pitem}{\item[$\diamond$]}

\newcommand{\tl}{\textasciitilde{}}
\newcommand{\typdiff}{\J{\textbackslash}}

\maketitle

\TODO{lean uses the term "metavariable" for our concept of "placeholder variable". Should we conform to Lean's terminology?}

\TODO{Metavariable term: the variable is meta in the sense that its interpretation is chosen by the metaprogram.
Even though it's actually still a variable in the object language.}

\TODO{Mention how reconstruction of recursive types is the last hurdle to fully extrinsic types}

\TODO{if recursive types can be reconstructed than so can everything else}

\TODO{inductive reasoning, in the general sense, consists of learning sufficient conditions P that enable deriving Q }

\TODO{reasoning by cases consists of deriving P implies Q 
by ensuresing P' implies Q for every P' that implie P
}

\TODO{mathematical induction is a particular form of reasoning by cases
where P is recursively defines, so P' resembles it, and for any P' that resembles P,
we can assume that P' -> Q', where Q' resembles Q.
}

\section{Introduction}
\label{sec:introduction}

Type inference in practical programming languages has historically relied on intrinsic types.
An expression has an intrinsic type if it is annotated with a type, such that the type
constrains the control flow of the program.  
Intrinsic types can occur in at least three ways: by annotating
expressions, constructions, or patterns.     
For example, the Standard ML expression \code{fn x => (f x)}
could be annotated with \code{val f' : int -> int = fn x => (f x)},
which artificially prunes the allowed control flow of expressions that use \code{f'}.
As for constructions and sums of patterns, consider a Standard ML program that computes the length of a list,
using \code{nil} and \code{cons} as data constructors representing lists, 
and using \code{zero} and \code{succ} as data constructors representing natural numbers.

\examp{
  \J{fun len(nil) = zero}
  \\
  \J{\ \ | len(cons(x,xs)) = succ(len(xs))}
}

\noindent
Standard ML uses the datatype mechanism to specify intrinsic types on both patterns and structures.

\examp{
  \J{datatype nat = zero | succ of nat} 
  \\
  \J{datatype 'a list = nil | cons of 'a * 'a list}
}

\noindent
For the constructions with \code{zero} and \code{succ}, the nat specification 
means that the weakest type the constructions could belong to is \code{nat},
which forbids mixing the constructions with those not specificated in the datatype, like \code{zero} with \code{nil}: 

\examp{
  \J{fn nil => zero}
  \\
  \J{\ | cons(x,xs) => nil}
}

\noindent
As for the patterns with \code{nil} and \code{cons}, the list specification 
means that the strongest assumed for the sum of the patterns
is \code{'a list}. 
Thus, it forbids pattern matching fewer patterns than those in the datatype.

\examp{
  \J{fun nlen(nil) = zero}
}

\noindent
Additionally, the list specification 
means that the strongest assumption allowed for \code{cons} is that its
content has type \code{'a * 'a list}. Thus, 
it forbids overspecifying pattern contents,
like a length function that only works on even length lists.  

\examp{
  \J{fun evlen(nil) = zero}
  \\
  \J{\ \ | evlen(cons(x,cons(y,zs))) = succ(succ(evlen(zs)))}
}

\noindent
With the datatype being both the strongest type of a sum of patterns and also the weakest
type of each construction, there is no additional loss in expressivity by
requiring that the weakest type of a sum of patterns to be the datatype, 
or inferring the strongest type of each construction to be the datatype.


Even with all the intrinsic specifications required for structural expressions, Standard ML,
and languages like it, ease the burden on the programmer by leveraging Hindley-Milner
type inference \cite{}, which can fairly general predicative types of 
non-structural expressions extrinsically, without intrinsic specifications. 
Despite the mitigating type inference capabilities of intrinsically typed systems, 
much more software these days is written
in untyped scripting languages, like Python and Javascript. 
These programs are not written with any control flow restrictions in mind,
and the are unlikely to conform to the control flow restrictions imposed by intrinsic types.
In fact, the lack of control flow restrictions may be one of the reasons that people
choose scripting languages over intrinsically typed languages.
Without artificial restrictions, programers can write code with a high degree of reusability,
with very little effort.
For instance, one might want a comparison function that works on both numbers and strings,
so that the caller can sort lists of numbers or lists of strings without having to
clutter the call site with a particular comparison function.  
Some have tried reconcile intrinsic types with code reusability by adding
advanced construct to languages, classes and type classes.
Indeed, these constructs do enable reusability, but they require additional
complexity and may fragment code in ways that obscures its essence.   
An alternative approach is to simply relax the intrinsic type requirements,
and infer types extrinsically according to the control flow of the program. 

There have been some relaxations of intrinsic types to enable sightly more code reusability.
Refinement types (Pfenning) \cite{} removed the intrinsic strongest type of sums of patters,
allowing pattern matching with incomplete pattern coverage. It inferred stronger
types too, by using intersection against the intrinsic weakest type to find safe but stronger types
that could be used in functions with stronger assumptions. 
Unfortunately, refinements aren't sufficient to handle the full range of structural mixing 
and code reusability allowed by untyped languages.
More recently, SuperF \cite{} demonstrated removal the the weakest intrinsic type,
which enabled it infer principal types with subtyping constraints, 
where second-order parametric
types were insufficient. However, SuperF was not concerned with how functions
with mixed structures could be represented literally.
On the other hand, MLstruct \cite{} consisted of a detailed language design that allows
constructing functions that mix structures, as long as information about the constructors
are specified in particular ways. 
MLStruct relaxed intrinsic requirements in a few ways, but still required a heavy annotation
burden and restricted control flow in a significant way.

This work tackles these limitations, in order to achieve a language design and type inference system,
such that there are almost no annotations are required, code reusability is uninhibited, and error-free programs
are guaranteed. The system is extremely extrinsic, requiring no annotations on structures.  
Despite the lack of assistance from intrinsic types, our system appears to be able to admit
nearly all the programs admitted by MLStruct and SuperF.
Moreover, our system's lack of certain intrinsic restrictions, which exists in MLstruct, 
clears the way for automatic reasoning by induction, which allows even greater code reusability
without sacrificing safety.  

\TODO{summarize contributions}
\TODO{full extrinsic, no annotations required to very programs on par with untyped pgograms in terms of code reusability}
\TODO{Normalizing all approximate control flow information into subtyping, allowing types to be infered in both directions}
\TODO{inductive reasoning on scalars without any intrinsic specifications}

\section{Overview}
\label{sec:overview}
We designed the type system around a simple applicative language with pattern matching,
since such a language is syntactically close to predicate logic. Likewise,
we designed the language of types to share many similarities with predicate logic. 
The extrinsic type system is able to infer very precise types by analyzing the control
flow of program in the form of generating and solving subtyping constraints. 
In many cases the type system can infer the weakest possible types of parameters,
and the strongest possible (i.e. most general or principal) types of results,
all without inhibiting the reusability the code.
To get a feel for the languages and reasoning capabilities, it's best to start by
looking at a few examples.

\Par{Finite Isomorphism.}
A function can be constructed from multiple paths. 
Consider the function \ms{\J{talky}}, which completes a simple English phrase.

\examp{
  \J{def talky = (}
  \\
  \I \J{[hello;@ => world;@]}
  \\
  \I \J{[good;@ => morning;@]} 
  \\
  \I \J{[thank;@ => you;@]} 
  \\
  \J{)}
}

\noindent
Since there is no recursion, the number of paths is finite.
For these simple finite functions, the type system constructs
an isomorphic type without any weakening in precision. 

\examp{
  \J{(<hello>@ -> <world>@) \&}
  \\
  \J{(<good>@ -> <morning>@) \&} 
  \\
  \J{(<thank>@ -> <you>@)} 
}

\noindent
Each path is represented using the path type,
and the paths are joined together using the intersection type.

\Par{Selection.}
Although the type of function $\J{talky}$ is no weaker than the actual function,
lifting into the type syntax avails it to approximate reasoning at application.
Consider an application of the function to a variable argument. 

\examp{
  \J{def x : <hello> @ | <thank> @ = ... in}
  \\
  \J{...}
  \\
  \J{talky(x)} 
  \\
  \J{...}
}

\noindent
If the type system affirms the type annotation,
then it goes on to infer the type of the application. 

\examp{
  \J{<world> @ | <you> @}
}

\noindent
In effect, the type system applies application to all possible arguments
and unions the results derived from only the paths of those arguments. 

\Par{Learning.}
The paths of a function are not always syntactically apparent. 
If a variable is applied to arguments, the variable should implicitly behave 
like a function, or else the program is erroneous. 
Consider the higher order function $\J{foo}$ which takes a function as input
and applies it to the number two and the boolean true.

\examp{
  \J{def foo = [f => (f(succ;succ;zero;@), f(true;@))]}
}

\noindent
Type systems for System F have traditionally struggled to find the best
type for such programs. If restricted to unqualified parametric types, there are many choices,
without there being a clear best or principal type, 
which is because it's not possible to find the weakest type for the function parameter
$\J{f}$.
One possibility is that the type system requires that $\J{f}$ is a function that wraps its
input in an option. 

\examp{
  \J{f : ALL[X] X -> <some> X}
}

\noindent
Another possibility is that the type system requires that $\J{f}$ is a function that wraps its
input in an list. 

\examp{
  \J{f : ALL[X] X -> (LFP[R] <nil> @ | <cons> X*R)}
}

\noindent
Both are valid, but neither is weaker than the other. The problem is that unqualified parametric
types are insufficient for describing the weakest safe input type.
It was not until SuperF introduced a technique for solving subtyping constraints
that a principal type could be automatically constructed for functions like $\J{foo}$
Our extrinsic type system relies on similar principles of solving subtyping constraints,
and likewise, can construct the weakest safe type for $\J{f}$, and thus the principal type for $\J{foo}$.

\examp{
  \J{ALL[X Y] (<succ> <succ> <zero> @ -> X) \& (<true> @ -> Y) -> X*Y}
}

\noindent
Although the subtyping constraints are not present in the type's notation,
the intersection of paths is chosen because it represents the weakest type
consistent with the subtyping constraints on a variable representing the parameter's type. 
Suppose the type system chooses the variable type $\J{F}$ for the input parameter $\J{f}$.
Eventually, due to the applications of the parameter $\J{f}$, the type
system will attempt to find a solution to two constraints on $\J{F}$

\examp{
  \J{F} \subtypes \J{(<succ> <succ> <zero> @ -> X)}
  \\
  \J{F} \subtypes \J{(<true> @ -> Y)}
}

\Par{Factorization.}
A function can represent infinite paths by using recursion.
Consider the function $\J{repeat}$, which takes an item and a natural number
as inputs and returns a list of that item repeated the input number of times.

\examp{
  \J{def repeat = [x => loop([self => [}
  \\
  \I \J{[zero;@ => nil;@]}
  \\
  \I \J{[succ;n => <cons>(x,self(n))]}
  \\
  \J{])]}
}

\noindent
One possibility for constructing the type of the recursive function would
be to follow the idea of finite isomorphism and construct a type 
that preserves all the paths. If our type system had a greatest fixed point
type constructor, then we could concisely represent such a type with minimal
rewriting.

\examp{
  \J{ALL[T] T -> GFP[R](} 
  \\
  \I \J{(<zero> @ -> <nil> @) \&}
  \\
  \I \J{(ALL[N L](R <: N->L) <succ> N -> <cons> T*L)}
  \\
  \J{)} 
}

\noindent
Although the strength of precision of this type would be very beneficial,
reasoning with such a type would be complicated. To keep the type system as
simple as possible for the purpose of extrinsic type inference with a high-degree
of code reusability, we leave such relational reasoning of recursive and co-recursive types out of
the type system. 
In this work, the type system actually avoids constructing a relational type, and instead, it
constructs a type that is significantly weaker than the actual behavior of the function,
by factoring the paths into a recursive space of inputs and a recursive space of outputs. 

\examp{
  \J{ALL[T] T ->} 
  \\
  \I \J{(LFP[R] <zero> @ | <succ> R)}
  \\
  \I \J{->}
  \\
  \I \J{(LFP[R] <nil> @ | <cons> T*R)}
}

\noindent
This eager weakening approximation enables the type system
to simply reason with recursive scalars and avoids the need for complicated
relational reasoning. In a follow up paper,
we will show how to remedy this limitation by augmenting the type system with
mechanisms for inductive relational reasoning.

\Par{Scalar Inflation.} 
Since the inputs to functions may have recursive types, the type 
system must be able to check for errors at function application by unrolling
recursive types.
Consider applying the function $\J{repeat}$ to some item and the number two, 
which results in the type system checking that two is in fact an instance of the natural numbers.

\examp{
  \J{ <succ> <succ> <zero> @} \subtypes \J{(LFP[R] <zero> @ | <succ> R)}
}

\noindent
The type system confirms that two is a valid input by unrolling the recursive type three times.
Unfortunately, because the type system eagerly factored the type of $\J{repeat}$, it
is unable to infer a stronger type for the application than the original list type. 
In the followup paper, inductive relational reasoning will remedy this limitation.


\Par{Scalar Induction.} 
Another scenario that may arise is one in which the input to a function has an recursive type.
Consider the application of the function $\J{repeat}$ to a variable whose type represents the even natural numbers.

\examp{
  \J{def x : (LFP[R] <zero> @ | <succ> <succ> R) = ... in}
  \\
  \J{...}
  \\
  \J{repeat(...)(x)} 
  \\
  \J{...}
}

\noindent
The application results in the type system checking that natural numbers are weaker than even numbers. 

\examp{
  \J{LFP[R] <zero> @ | <succ> <succ> R}
  \\
  \subtypes
  \\
  \J{LFP[R] <zero> @ | <succ> R}
}

\noindent
Since even numbers are represented recursively, the type system performs mathematical induction,
proving under the assumption that the subtyping relation holds for the self reference of the even numbers. 

\examp{
  \J{E} \subtypes \J{(LFP[R] <zero> @ | <succ> R)}
  \\
  \entails
  \\
  \J{(<zero> @ | <succ> <succ> E)} \subtypes \J{(LFP[R] <zero> @ | <succ> R)}
}

\noindent
In practice, the type system actually substitutes the supertype into
the self reference of the subtype, if the self reference occurs in a positive position of the subtype.

\examp{
  \J{(<zero> @ | <succ> <succ> (LFP[R] <zero> @ | <succ> R))} 
  \\
  \subtypes 
  \\
  \J{(LFP[R] <zero> @ | <succ> R)}
}

\noindent
In effect, by using mathematical induction, the type system ensures that 
every type contained within the type of even natural numbers is also
contained within the type of natural numbers.
In followup work, we will extend this inductive reasoning to handle inductive relations. 

\Par{Path Sensitivity} 
Functions on lists can be generic, in that they may be independent of the item values, as in \code{repeat}.
Additionally, they may factor out the details that depend on item values, as in \code{sort}. 

\examp{
  \J{def sort : ALL[T] (T*T->} \textbf{Bool} \J{) -> }\textbf{List}(\J{T}) \J{ -> } \textbf{List}(\J{T}) \J{ = }
  \\
  \J{[ cmp => ...] in ...}
}

\noindent
The notation \ms{\textbf{Bool}} and \ms{\textbf{List}(\J{X})} are meta-language terms representing types.
The function $\J{sort}$ can be specialized by passing in different implementations of comparison functions.
Suppose we already have one scalar comparison function and one lexicographic comparison function.

\examp{
  \J{def scalarCmp : }\textbf{Nat} \J{ -> } \textbf{Nat} \J{ -> } \textbf{Bool} \J{ = ... in}
  \\
  \J{def lexicoCmp : }\textbf{List}(\textbf{Nat}) \J{ -> } \textbf{List}(\textbf{Nat}) \J{ -> } \textbf{Bool} \J{ = ... in} 
  \\
  \J{...} 
}

\noindent
In Standard ML, the programmer would need a different specialization for every predefined
datatype. 

\examp{
  \J{def scalarSort = sort(scalarCmp) in}
  \\
  \J{def lexicoSort = sort(lexicoCmp) in} 
  \\
  \J{...} 
}

\noindent
Having many different specializations burdens the client of
the generic function to know many different names for roughly the same behavior and can obscure the meaning
of code at the call site. 
Untyped languages or extrinsically typed languages can bypass this rigidity found in 
intrinsically typed languages by mixing the specialized procedures into a single generalized procedures.

\examp{
  \J{def genCmp = [a,b => } 
  \\
  \I \J{a |> (}
  \\
  \I\I \J{[zero;@ => scalarCmp(a,b)]}
  \\
  \I\I \J{[succ;n => scalarCmp(a,b)]}
  \\
  \I\I \J{[nil;@ => lexicoCmp(a,b)]}
  \\
  \I\I \J{[cons;(x,xs) => lexicoCmp(a,b)]}
  \\
  \I \J{)}
  \\
  \J{] in} 
  \\
  \J{...} 
}

\noindent
Extrinsic type inference allows this mixing by inferring types locally for each path,
such that the type inferred from one path is not required for another.
As such, a generalized comparison procedure may be constructed and the
programmer can use the same sort function no matter if its inputs are scalars or vectors.

\examp{
  \J{def genSort : (TOP}
  \\
  \I \J{\& (}\textbf{List}(\textbf{Nat}) \J{ -> } \textbf{List}(\textbf{Nat})
  \\
  \I \J{\& (}\textbf{List}(\textbf{List}(\textbf{nat})) \J{ -> } \textbf{List}(\textbf{List}(\textbf{Nat}))
  \\
  \J{) = sort(genCmp)}
}

\noindent
This particular form of type inference, known as occurrence typing \cite{},
has already been studied and used to great effect in type inference for Python and Javascript.
Additionally, the path sensitivity of type inference obviates the need 
for destruction in pattern matching.  
For example, the \code{repeat} function can be rewritten.

\examp{
  \J{def repeat = [x => loop([self => [n =>}
  \\
  \I\J{n |> [}
  \\
  \I\I \J{[zero;\_ => nil;(n.zero)]}
  \\
  \I\I \J{[succ;\_ => cons;(x,self(n.succ))]}
  \\
  \I \J{]}
  \\
  \J{]])]}
}

\noindent
Thus, programs in Python and Javascript, which only having conditional branching,
and no pattern matching, can rely on path sensitivity to simulate pattern matching. 

\Par{Pruning} 

\TODO{...}

\Par{Streaming} 

\TODO{...}

\examp{
  \J{loop([self => [x =>}
  \\
  \I\J{[@ => (x, self(succ;x))]}
  \\
  \J{]]}
}

\TODO{...}



\examp{
  \J{ALL[T] T -> @ ->}
  \\
  \I\J{LFP[U] (LFP[R] T | <succ> R) * (@ -> U)}
}

% (ALL[G009] (G009 -> (LFP[G007] (EXI[G010] (G010 <: G007)  : (@ -> ((LFP[G007] (<succ> G007 | G009)) * G010))))))

\Par{Subtraction} 

\TODO{...}

\examp{
  \J{LFP[R] (<zero> @) | (<succ> <succ> R)}
  \\
  \subtypes
  \\
  \J{TOP\typdiff (<succ> <succ> <succ> <zero> @)}
}

\section{Dynamic System}
\label{sec:dynamic_system}

\TODO{factor out definitions common to both dynamic and static systems into basic system}

\TODO{flip records, functions and other collections to associate right (matching Lean design)}

\noindent
To demonstrate the capabilities of our extrinsic type system, 
we rely on an applicative programming language with pattern matching.
The design attempts to keep the forms of expressions and types relatively 
similar to each other when their semantics are also similar. 
Although the expression language is based on pattern matching, 
the type inference techniques apply just as well to languages
that are based on conditional branching and destructors instead of pattern matching.
In fact, conditional branching and destructors can be encoded from pattern matching,
and used without any loss in type inference precision.

\begin{definition}
  \label{def:expression}
  \nopad 
  \small
  \[\begin{array}[t]{rr@{}l}
    \com{expression} &
    \nonterm{e} 
    & 
    \is 
    \begin{array}[t]{@{}l}
      x \sep 
      \J{@} \sep
      r \sep
      f \sep 
      e\J{.}l \sep
      e\J{(}e\J{)} \sep
      \J{loop(}e\J{)} \sep
      \J{def}\ x\ \J{:}\ \tau\ \J{=}\ e\ \J{in}\ e \sep
      s
    \end{array}
    \\
    \com{record} &
    \nonterm{r} & \is \epsilon \sep r\ \J{<}l\J{>}e 
    \\
    \com{function} &
    \nonterm{f} & \is \epsilon \sep f\ \J{[}p\J{=>}e \J{]}
    \\
    \com{sugar} &
    \nonterm{s} 
    & 
    \is 
    \begin{array}[t]{@{}l}
      l \J{;} e \sep
      e \J{,} e \sep
      e\ \J{|>}\ e \sep
      \J{if}\ e\ \J{then}\ e\ \J{else}\ e \sep
      \J{def}\ x\ \J{=}\ e\ \J{in}\ e
    \end{array}
  \end{array}\]
\end{definition}

\noindent
An expression is a variable \ms{x}, unit \code{@},
a record \ms{... \J{<}l\J{>} e ...}, a function \ms{...\J{[}p \J{=>} e \J{]}...}, 
a projection \ms{e\J{.}l}, an application \ms{e\J{(}e\J{)}}, 
a loop \ms{\J{loop(}e\J{)}},
an annotation \ms{\J{def}\ x\ \J{:}\ \tau\ \J{=}\ e\ \J{in}\ e},
or sugar.
These forms are similar to those seen in other applicative structural languages,
with a few choices that are a bit less common. One thing that may be
a bit unconventional is that a function may syntactically consist of 
multiple paths guarded by patterns.  

\begin{definition} 
  \label{def:expression_desugaring}
  \emph{expression desugaring}
  \hfill
  \small
  \boxed{\llbracket s \rrbracket^\square = e}
  \nopad
  \begin{align*}
      \llbracket l \J{;} e \rrbracket^\square &= \J{<}l \J{>}\ e
      \\
      \llbracket e_l \J{,} e_r \rrbracket^\square &= \J{<left>}\ e_l \ \J{<right>}\ e_r
      \\
      \llbracket e_l\ \J{|>}\ e_r \rrbracket^\square &= e_r\J{(}e_l\J{)}
      \\
      \llbracket \J{if}\ e_c\ \J{then}\ e_t\ \J{else}\ e_f \rrbracket^\square &= 
      \begin{array}[t]{@{} l}
        e_c \J{ |> (}
          \\
          \I \J{[true;@ => } e_t \J{]}
          \\
          \I \J{[false;@ => } e_f \J{]}
          \\
        \J{)}
      \end{array}
      \\
      \llbracket \J{def}\ x\ \J{=}\ e\ \J{in}\ e' \rrbracket^\square &= \J{[}x \J{=>} e'\J{]}\J{(}e\J{)}
  \end{align*}

\end{definition} 

\noindent
For convenience and clarity, there is sugar for 
singleton records \ms{l\J{;}e},
tuples \ms{e\J{,}e}, 
unannotated bindings \ms{\J{def}\ x\ \J{=}\ e\ \J{in}\ e}, flow \ms{e\ \J{|>}\ e},
and conditional branching \ms{\J{if}\ e\ \J{then}\ e\ \J{else}\ e}.
For any metalanguage statement or term containing sugar $s$, we assume that $s$ is implicitly replaced with $\llbracket s \rrbracket^\square$.

\begin{definition}
  \label{def:pattern}
  \small
  \nopad
  \[\begin{array}[t]{rr@{}l}
    \com{pattern} &
    \nonterm{p} & \is 
      x \sep
      \J{@} \sep
      k
    \\
    \com{record} &
    \nonterm{k} & \is \epsilon \sep k\ \J{<}l\J{>}p
  \end{array}\]
\end{definition}

\noindent
A pattern is either unit or a record consisting of patterns.

\begin{definition}
  \label{def:value}
  \small
  \nopad
  \[\begin{array}[t]{rr@{}l}
    \com{value} &
    \nonterm{v} & \is 
      \J{@} \sep
      g \sep
      f
    \\
    \com{record} &
    \nonterm{g} & \is \epsilon \sep g\ \J{<}l\J{>}v
  \end{array}\]
\end{definition}

\noindent
A value may be unit, a function, or a record consisting of only values. 

\begin{definition}
  \label{def:progression}
  \emph{progression}
  \hfill 
  \small
  \boxed{e \rightsquigarrow e}
  \nopad
  \begin{mathpar}
    \inferrule {
      e \rightsquigarrow e'
    } {
      r\ \J{<}l\J{>}e \rightsquigarrow r\ \J{<}l\J{>}e'
    }

    \inferrule {
      r \rightsquigarrow r'
    } {
      r\ \J{<}l\J{>} v \rightsquigarrow r'\ \J{<} l \J{>} v
    }

    \inferrule {
      e \rightsquigarrow e'
    } {
      e\J{.}l \rightsquigarrow e'\J{.}l
    }

    \inferrule {
      \J{<}l\J{>}v \in g
      \\
      \forall\ e .\ \J{<}l\J{>}e \in g \implies e = v
    } {
      g\J{.}l \rightsquigarrow v
    }

    \inferrule {
      e_f \rightsquigarrow e_f'
    } {
      e_f\J{(}e\J{)} \rightsquigarrow e_f'\J{(}e\J{)}
    }

    \inferrule {
      e \rightsquigarrow e'
    } {
      f\J{(}e\J{)} \rightsquigarrow f\J{(}e'\J{)}
    }

    \inferrule {
      f\J{(}v\J{)} \rightsquigarrow e'
      \\
      \textbf{fv}(e) \subseteq \textbf{fv}(p)
    } {
      (f\ \J{[}p\J{=>}e\J{]})\J{(}v\J{)} \rightsquigarrow e' 
    }

    \inferrule {
      \nexists e' .\ f\J{(}v\J{)} \rightsquigarrow e'
      \\
      v  \simeq p \given \sigma 
    } {
      (f\ \J{[}p\J{=>}e\J{]})\J{(}v\J{)} \rightsquigarrow e[\sigma]
    }

    \inferrule {
      e \rightsquigarrow e'
    } {
      \J{loop(}e\J{)} \rightsquigarrow \J{loop(}e'\J{)}
    }

    \inferrule {
    } {
      \J{loop([}x\J{=>}e\J{])} \rightsquigarrow e[x \slash \J{loop([}x\J{=>}e\J{])}]
    }
  \end{mathpar}
\end{definition}

\noindent
Progression of an expression is a fairly typical small-step operational semantics. 
Its most atypical aspect is handling the unconventional multi-path syntax of
functions.
Since a function may consist of multiple paths guarded by patterns, 
application subsumes pattern matching.
Recursion is enabled by the \code{loop} keyword, such that
\code{loop} behaves like a fixed point combinator.


\begin{definition}
  \label{def:value_map}
  \small
  \nopad
  \[\begin{array}[t]{rr@{}l}
    \com{value map} &
    \nonterm{\sigma} & \is \epsilon \sep \sigma\ x \slash v 
  \end{array}\]
\end{definition}

\noindent
A value map assigns variables to values.
Progression relies on interpretation maps to collect
values that are relevant from pattern matching \ms{v \simeq p \given \sigma},
and then to use them in a subsequent expression via substitution 
\ms{e[\sigma]}


\begin{definition}
  \label{def:type}
  \small
  \nopad
  \[\begin{array}[t]{rr@{}l}
    \com{type} &
    \nonterm{\tau} & \is 
    \begin{array}[t]{@{}l}
      \alpha \sep
      \J{@} \sep
      \kappa \sep 
      \phi \sep 
      \psi \sep 
      \J{LFP[}\alpha\J{]}\tau \sep
      \tau \typdiff \eta \sep
      \gamma
    \end{array}
    \\
    \com{implication} &
    \nonterm{\kappa} & \is 
    \begin{array}[t]{@{}l}
      \J{<} l \J{>} \tau \sep 
      \tau\J{->}\tau
    \end{array}
    \\
    \com{expansion} &
    \nonterm{\phi} & \is 
    \begin{array}[t]{@{}l}
      \tau\J{|}\tau \sep 
      \J{EXI[}\Theta\J{]} \Delta \J{:} \tau
    \end{array}
    \\
    \com{refinement} &
    \nonterm{\psi} & \is 
    \begin{array}[t]{@{}l}
      \tau\J{\&}\tau \sep 
      \J{ALL[}\Theta\J{]} \Delta \J{:} \tau
    \end{array}
    \\
    \com{subtractable} &
    \nonterm{\eta} & \is 
    \begin{array}[t]{@{}l}
      \J{EXI[}\Theta \J{]} \eta \sep 
      \rho
    \end{array}
    \\
    \com{pattern} &
    \nonterm{\rho} & \is 
    \begin{array}[t]{@{}l}
      \alpha \sep
      \J{@} \sep
      \J{<}l\J{>}\rho \sep 
      \rho\J{\&}\rho \sep
      \rho\J{*}\rho
    \end{array}
    \\
    \com{sugar} &
    \nonterm{\gamma} & \is 
    \begin{array}[t]{@{}l}
      \J{TOP} \sep \J{BOT} \sep 
      \tau \J{*} \tau 
    \end{array}
  \end{array}\]
\end{definition}

\noindent
Types provide the representation for approximate specification.
To accommodate a high degree of structure reusability,
it is important that the forms of types accommodate a wide range of precision.
To manage the complexity of the various forms, we subdivide the type forms
according to some themes.
A type \ms{\tau} may consist of unit \code{@}, variables \ms{\alpha}, 
implications, expansions, refinements,
least fixed points \ms{\J{LFP[}\alpha \J{]} \tau},
differences \ms{\tau \typdiff \eta}, and sugar.


An implication \ms{\kappa} may take the form of  
an entry \ms{\J{<}l\J{>}\tau} or a path \ms{\tau \J{->} \tau},
reflecting the structure of its inhabitants.
An expansion \ms{\phi} is either a union \ms{\tau \J{|} \tau} or 
an existential \ms{\J{EXI[}\Theta \J{]} \Delta \J{:} \tau},
representing weakenings of their constituent types. 
An existential is parameterized by type variables \ms{\Theta}
and positively qualified by subtyping constraints \ms{\Delta}, 
such that the weaker the qualification, the weaker the existential.
A refinement \ms{\psi} is either an intersection \ms{\tau \J{\&} \tau} or 
a universal \ms{\J{ALL[} \Theta \J{]} \Delta \J{:}\tau},
representing strengthenings of their constituent types. 
An universal is parametrized by type variables \ms{\Theta} 
and negatively qualified by subtyping constraints $\Delta$,
such that the weaker the qualification, the stronger the universal.


A least fixed point \ms{\J{LFP[}\alpha \J{]}}
represents an infinite space of structures, 
such as natural numbers or lists. 
A difference \ms{\tau \typdiff \eta} is a space without inhabitation by certain structures.
The subtracted part is restricted to a subtractable form,
which is a form that our system can decide the inhabitation
assuming additional semantics restrictions.


\begin{definition} 
  \label{def:type_desugaring}
  \emph{type desugaring}
  \hfill 
  \small
  \boxed{\llbracket \gamma \rrbracket^\square = \tau}
  \nopad
  \begin{align*}
    \llbracket \J{TOP} \rrbracket^\square &= \J{EXI[} \alpha \J{]} \alpha
    \\
    \llbracket \J{BOT} \rrbracket^\square &= \J{ALL[} \alpha \J{]} \alpha
    \\
    \llbracket \tau_l \J{*} \tau_r \rrbracket^\square &= (\J{<left>}\ \tau_l)  \J{\&} (\J{<right>}\ \tau_r)
  \end{align*}
\end{definition} 



\noindent
For convenience and clarity, types include syntactic sugar for top $\J{TOP}$, 
bottom $\J{BOT}$, and pairs $\tau \J{*} \tau$.
For any metalanguage statement or term containing sugar $\gamma$, we assume that $\gamma$ is implicitly replaced with $\llbracket \gamma \rrbracket^\square$.

\begin{definition}
  \label{def:sequence}
  \small
  \nopad
  \[\begin{array}[t]{rr@{}l}
    \com{type var seq} &
    \nonterm{\Theta} & \is \epsilon \sep \Theta\ \alpha 
    \\
    \com{subtyping seq} &
    \nonterm{\Delta} & \is \epsilon \sep \Delta \  \tau \J{<:} \tau
    \\
    \com{typing seq} &
    \nonterm{\Gamma} & \is \epsilon \sep \Gamma\ x \hastype \tau
    \\
    \com{locale seq} &
    \nonterm{\Pi} & \is \epsilon \sep \Pi\ (\Theta, \Delta, \tau) 
    \\
    \com{subtractable type seq} &
    \nonterm{\Xi} & \is \epsilon \sep \Xi\ \eta 
    \\
    \com{negatable subtyping seq} &
    \nonterm{\Omega} & \is \epsilon \sep 
      \Omega\ \tau \J{<:} \eta \sep 
      \Omega\ \alpha \J{<:} \tau
    \\
    \com{composite seq} &
    \nonterm{C} & \is \epsilon \sep C\ \iota 
    \\
    \com{item} &
    \nonterm{\iota} & \is 
    \alpha
    \sep 
    \tau \J{<:} \tau
    \sep
    x \hastype \tau
    \sep
    (\Theta, \Delta, \tau)
    \sep
    \eta
    \sep
    \tau \J{<:} \eta

  \end{array}\]
\end{definition}


\noindent
There are various scenarios where objects should be collected into in sequences or sets.
For metalanguage operations or predicates that are generic with respect the form of a collection's items,
it is convenient to have a composite sequence.
A sequence is either empty \ms{\epsilon} or an extension,
such as \ms{C\ \iota} in the case of the composite sequence. 
For clarity, we omit writing \ms{\epsilon} explicitly when describing semantics or instances of the forms.
That is, for any structure, if there is a malformed structure due to a missing piece, 
we assume it is replaced by a well-formed structure, 
where \ms{\epsilon} is inserted wherever necessary to make it well-formed.


\begin{definition} 
  \label{def:sequence_setting}
  \emph{sequence setting}
  \hfill 
  \small
  \boxed{\textbf{set}(C) = S}
  \nopad
  \[
  \begin{array}[t]{r@{}c@{}l}
      \textbf{set}(\epsilon)
      &\ =\ & 
      \emptyset 

      \\

      \textbf{set}(C\ \iota)
      &\ =\ & 
      \textbf{set}(C) \cup [ \iota ] 
  \end{array}
  \]
\end{definition} 

\noindent
In order to leverage set arithmetic and its theory,
we implicitly convert sequences into sets. 
We assume that any sequence \ms{C} may be implicitly replaced with \ms{\textbf{set}(C)}.

\begin{definition} 
  \label{def:set_sequencing}
  \emph{set sequencing}
  \hfill 
  \small
  \boxed{\textbf{seq}(S) = C}
  \nopad
  \\
  \[
  \begin{array}[t]{r@{}c@{}l}
      \textbf{seq}(\emptyset)
      &\ =\ & 
      \epsilon 

      \\

      \textbf{seq}(S \cup [\iota])
      &\ =\ & 
      \textbf{seq}(S \backslash [\iota])\ \iota 
  \end{array}
  \]
\end{definition} 

\noindent
Likewise, it also beneficial to convert sets back into sequences, to leverage
definitions that decompose sequences according to their structures. 
We assume that any set \ms{S} may be implicitly replaced with \ms{\textbf{seq}(S)}.

\begin{definition}
  \label{def:type_map}
  \small
  \nopad
  \[\begin{array}[t]{rr@{}l}
    \com{type map} &
    \nonterm{\delta} & \is \epsilon \sep \delta\ \alpha \slash \tau
  \end{array}\]
\end{definition}

\noindent
A type map assigns variables to types.
Since a type may be constructed from variables, it 
is necessary to specify an interpretation of the variables
when judging the inhabitation of the type.
Thus, typing relies on type maps to provide interpretations
for type variables.

\begin{definition}
  \label{def:valuing}
  \emph{valuing}
  \hfill
  \small
  \boxed{\downharpoonleft e}
  \nopad
  \begin{mathpar}
    \inferrule { 
      \TODO{...}
    } {
      \downharpoonleft 
      e 
    } 
  \end{mathpar}
\end{definition}

\begin{definition} 
  \label{def:type_sub_fold}
  Type Substitution Fold 
  \hfill
  \small
  \boxed{
    \tau [\alpha \slash \tau']^n
  }
  \\
  \[
  \begin{array}[t]{rclcl}
      \tau [\alpha \slash \tau']^0 
      &=& 
      \tau' 
      \\
      \tau [\alpha \slash \tau']^{(n + 1)} 
      &=& 
      \tau [\alpha \slash \tau[\alpha \slash \tau']^n]
  \end{array}
  \]
\end{definition}


\TODO{create a seperate relation restricted negations of model typing}

\TODO{use antec rules to compensate for non-self reference in negative position}

\TODO{consider renaming to dynamic typing}

\TODO{completely redo model typing with well founded recursion}

\TODO{define monotonicity/antitonicity/polarity check}

\TODO{create recursive definition of least fixed point check}

\TODO{use separate typing for interpreting variables as finite types}

\TODO{use non-syntactic description of implication}

\TODO{remove implication antecedent rules}

\TODO{lift progression rule out to highest level for monotonic recursion}

% \begin{definition}
%   \label{def:model_progress_typing}
%   \emph{model progressive typing}
%   \hfill
%   \small
%   \boxed{\delta \satisfies e \hastype^* \tau}
%   \nopad
%   \begin{mathpar}
%     \inferrule { 
%       \downharpoonleft e
%       \up
%       \delta \satisfies e \hastype \tau 
%     } {
%       \delta \satisfies e \hastype^* \tau 
%     } 

%     \inferrule { 
%       e \rightsquigarrow e'
%       \\
%       \delta \satisfies e' \hastype^* \tau 
%     } {
%       \delta \satisfies e \hastype^* \tau 
%     } 
%   \end{mathpar}

% \end{definition}

\begin{definition}
  \label{def:model_typing}
  \emph{model typing}
  \hfill
  \small
  \boxed{\delta \satisfies e \hastype \tau}
  \nopad
  \\\\
  \TODO{well founded just on type}
  \\\\
  \TODO{should vacuous universal be prevented}
  \[
  \begin{array}[t]{rclcl}
      \delta \satisfies e \hastype \alpha 
      &\iff& 
      \alpha \slash \tau \in \delta
      \up
      \VDash e \hastype \tau

      \\

      \delta \satisfies e \hastype \J{@}
      &\iff& 
      e \rightsquigarrow^* \J{@}

      \\

      \delta \satisfies e \hastype \J{<} l \J{>} \tau
      &\iff& 
      \forall e' \qua
      e\J{.}l \rightsquigarrow^* e' 
      \implies
      \delta \satisfies e' \hastype \tau 

      \\

      \delta \satisfies e \hastype \tau_l \J{->} \tau_r
      &\iff& 
      \begin{array}[t]{@{}l}
        \forall e_f, e_l, e_r \qua
        e \rightsquigarrow^* e_f
        \up
        e_f\J{(}e_l\J{)} \rightsquigarrow^* e_r
        \implies
        \delta \satisfies e_l \hastype \tau_l 
        \implies
        \delta \satisfies e_r \hastype \tau_r
      \end{array}

      \\

      \delta \satisfies e \hastype \tau_l \J{|} \tau_r 
      &\iff& 
      \delta \satisfies e \hastype \tau_l
      \down
      \delta \satisfies e \hastype \tau_r

      \\

      \delta \satisfies e \hastype \tau_l \J{\&} \tau_r 
      &\iff& 
      \delta \satisfies e \hastype \tau_l
      \up
      \delta \satisfies e \hastype \tau_r

      \\

      \delta \satisfies e \hastype \J{EXI[}\Theta\J{]}\Delta \J{:} \tau
      &\iff& 
      \exists \delta' \qua
      \textbf{dom}(\delta') \subseteq \Theta
      \up
      \delta \oplus \delta' \satisfies \Delta 
      \up
      \delta \oplus \delta' \satisfies e \hastype \tau

      \\

      \delta \satisfies e \hastype \J{ALL[}\Theta\J{]}\Delta \J{:} \tau
      &\iff& 
      \forall \delta' \qua
      \textbf{dom}(\delta') \subseteq \Theta
      \implies
      \delta \oplus \delta' \satisfies \Delta 
      \implies
      \delta \oplus \delta' \satisfies e \hastype \tau

      \\

      \delta \satisfies e \hastype \tau_l \typdiff \tau_r
      &\iff& 
      \delta \satisfies e \hastype \tau_l
      \up
      \delta \not\satisfies e \hastype \tau_r

      \\

      \delta \satisfies e \hastype \J{LFP[}\alpha\J{]}\tau 
      &\iff& 
      \alpha \wr^+ \tau
      \up
      \exists n \qua \VDash e \hastype (\tau[\alpha \slash \J{BOT}]^n)[\delta]

      % \\

      % \delta \satisfies e \hastype \tau 
      % &\iff& 
      % \VDash e \hastype \tau[\delta]

      % \\

      % \delta \satisfies r \hastype \J{<} l \J{>} \tau
      % &\iff& 
      % \J{<}l \J{>} e \in r 
      % \up
      % \delta \satisfies e \hastype \tau 

      % \\

      % \delta \satisfies \J{[}p \J{=>} e \J{]} \hastype \tau_l \J{->} \tau_r
      % &\iff& 
      % \forall \sigma \qua 
      % \delta \satisfies p[\sigma] \hastype \tau_l
      % \implies
      % \delta  \satisfies e[\sigma] \hastype \tau_r

      % \\

      % \delta \satisfies \J{[}p \J{=>} e \J{]}\ f\hastype \tau_l \J{->} \tau_r
      % &\iff& 
      % \begin{array}[t]{@{}ll}
      %   \nexists \sigma \qua \delta \satisfies p[\sigma] \hastype \tau_l
      %   & \textbf{if } \delta \satisfies f \hastype \tau_l \J{->} \tau_r
      %   \\
      %   \begin{array}[t]{@{}l}
      %   \forall \tau_p, \tau_f, e' \qua 
      %   \\
      %   \delta \VDash \J{[}p \J{=>} e \J{]} \hastype \tau_p
      %   \up
      %   \delta \VDash f \hastype \tau_f
      %   \implies
      %   \\
      %   \delta \satisfies e' \hastype \tau_l \J{->} \tau_r 
      %   \implies
      %   \delta \VDash e' \hastype \tau_p \J{\&} \tau_f
      %   \end{array}
      %   & \textbf{otherwise}
      % \end{array}
  \end{array}
  \]
\end{definition}

\begin{definition}
  \label{def:model_finite_typing}
  \emph{model finite typing}
  \hfill
  \small
  \boxed{\VDash e \hastype \tau}
  \nopad
  \begin{mathpar}
  \TODO{well founded on type}
  \end{mathpar}
\end{definition}

\begin{definition}
  \label{def:model_scratch_typing}
  \emph{model scratch typing}
  \hfill
  \small
  \boxed{\delta \satisfies e \hastype \tau}
  \nopad
  \begin{mathpar}
    % \inferrule { 
    %   \TODO{alt def}
    %   \\\\
    %   \forall \tau' \qua  
    %   \delta \satisfies \tau[\alpha\slash\tau'] \subtypes \tau'
    %   \implies
    %   \delta \satisfies e \hastype \tau'
    % } {
    %   \delta \satisfies e \hastype \J{LFP[} \alpha \J{]} \tau
    % } 

    % \inferrule { 
    %   \delta \ \alpha \slash \J{LFP[}\alpha \J{]} \tau \satisfies e \hastype \tau
    % } {
    %   \delta \satisfies e \hastype \J{LFP[} \alpha \J{]} \tau
    % } 


    % \inferrule { 
    %   \delta \satisfies e \hastype \tau_l
    % } {
    %   \delta \satisfies e \hastype \tau_l \J{|} \tau_r
    % } 

    % \inferrule { 
    %   \delta \satisfies e \hastype \tau_r
    % } {
    %   \delta \satisfies e \hastype \tau_l \J{|} \tau_r
    % } 


    % \inferrule { 
    %   \textbf{dom}(\delta') \subseteq \Theta
    %   \\
    %   \delta \oplus \delta' \satisfies \Delta 
    %   \\
    %   \delta \oplus \delta' \satisfies e \hastype \tau
    % } {
    %   \delta \satisfies e \hastype \J{EXI[}\Theta\J{]}\Delta \J{:} \tau
    % } 

    % \inferrule { 
    %   \delta \satisfies e \hastype \tau_l
    %   \\
    %   \delta \satisfies e \hastype \tau_r
    % } {
    %   \delta \satisfies e \hastype \tau_l \J{\&} \tau_r
    % } 

    % \inferrule { 
    %   \forall \delta' \qua
    %   \textbf{dom}(\delta') \subseteq \Theta \implies
    %   \delta \oplus \delta' \VDash \Delta \implies 
    %   \delta \oplus \delta' \satisfies e \hastype \tau
    %   \\\\
    %   \exists \delta' \qua 
    %   \textbf{dom}(\delta') \subseteq \Theta \up 
    %   \delta \oplus \delta' \satisfies \Delta
    % } {
    %   \delta \satisfies e \hastype \J{ALL[}\Theta\J{]}\Delta \J{:} \tau
    % } 

    % \inferrule { 
    %   \J{<}l \J{>} v \in g
    %   \\
    %   \delta \satisfies v \hastype \tau 
    %   \\
    %   \forall v' \qua \J{<} l \J{>} v' \in g \implies v' = v
    % } {
    %   \delta \satisfies g \hastype \J{<} l \J{>} \tau
    % } 

    % \inferrule { 
    %   \TODO{diff antec}
    %   \\\\
    %   p \Uparrow \eta 
    %   \\
    %   \delta \satisfies f \hastype \tau_l \J{->} \tau_r
    % } {
    %   \delta \satisfies \J{[}p \J{=>} e \J{]}\ f \hastype \tau_l \typdiff \eta \J{->} \tau_r
    % } 


    % \inferrule { 
    %   \TODO{needs separate inter antec}
    %   \\\\
    %   \forall \sigma \qua 
    %   \delta \VDash p[\sigma] \hastype \tau_l
    %   \implies
    %   \delta  \satisfies e[\sigma] \hastype \tau_r
    % } {
    %   \delta \satisfies \J{[}p \J{=>} e \J{]}\ f\hastype \tau_l \J{->} \tau_r
    % }

    % \inferrule { 
    %   \TODO{inter antec left}
    %   \\\\
    %   \delta \satisfies e \hastype \tau_l \J{->} \tau
    % } {
    %   \delta \satisfies e \hastype (\tau_l \J{\&} \tau_r) \J{->} \tau
    % }

    % \inferrule { 
    %   \TODO{inter antec right}
    %   \\\\
    %   \delta \satisfies e \hastype \tau_r \J{->} \tau
    % } {
    %   \delta \satisfies e \hastype (\tau_l \J{\&} \tau_r) \J{->} \tau
    % }

    % \inferrule { 
    %   \TODO{all antec}
    %   \\\\
    %   \textbf{dom}(\delta') \subseteq \Theta 
    %   \\
    %   \delta \oplus \delta' \satisfies \Delta 
    %   \\
    %   \delta \oplus \delta' \satisfies e \hastype \tau_l \J{->} \tau_r
    % } {
    %   \delta \satisfies e \hastype (\J{ALL[}\Theta \J{]}\Delta\ \tau_l) \J{->} \tau_r
    % }

    % \inferrule { 
    %   \TODO{union antec}
    %   \\\\
    %   \delta \satisfies e \hastype \tau_l \J{->} \tau
    %   \\
    %   \delta \satisfies e \hastype \tau_r \J{->} \tau
    % } {
    %   \delta \satisfies e \hastype (\tau_l \J{|} \tau_r) \J{->} \tau
    % }

    % \inferrule { 
    %   \TODO{exi antec}
    %   \\\\
    %   \forall \delta' \qua 
    %   \textbf{dom}(\delta') \subseteq \Theta
    %   \implies
    %   \delta \oplus \delta' \VDash \Delta 
    %   \implies
    %   \delta \satisfies e \hastype \tau_l \J{->} \tau_r
    % } {
    %   \delta \satisfies e \hastype (\J{EXI[} \Theta \J{]}\Delta\ \tau_l) \J{->} \tau_r
    % }

    % \inferrule { 
    %   \TODO{lfp antec}
    %   \\\\
    %   \alpha \notin \textbf{ftv}(\tau_l[\alpha\slash\J{@}]^+)
    %   \\
    %   \delta\ \alpha \slash (\J{LFP[}\alpha \J{]} \tau_l) \satisfies e \hastype \tau_l \J{->} \tau_r
    % } {
    %   \delta \satisfies e \hastype (\J{LFP[}\alpha \J{]} \tau_l) \J{->} \tau_r
    % }

    % \inferrule { 
    %   \delta \satisfies e \hastype \tau_l
    %   \\
    %   \neg\ (\delta \satisfies e \hastype \tau_r)
    % } {
    %   \delta \satisfies e \hastype \tau_l \typdiff \tau_r
    % } 

    % \inferrule { 
    % } {
    %   \delta \satisfies \J{@} \hastype \J{@}
    % } 

    % \inferrule { 
    %   \TODO{remove unflipped}
    %   \\\\
    %   \delta \satisfies f \hastype \tau_l \J{->} \tau_r
    % } {
    %   \delta \satisfies f\ \J{[}p \J{=>} e \J{]}\hastype \tau_l \J{->} \tau_r
    % } 

    % \inferrule { 
    %   \TODO{remove unflipped}
    %   \\\\
    %   \forall \sigma \qua
    %   \delta \satisfies p[\sigma] \hastype \tau_l
    %   \implies
    %   \delta  \satisfies e[\sigma] \hastype \tau_r
    %   \\\\
    %   \forall \tau_l', \tau_r' \qua
    %   \delta \satisfies f \hastype \tau_l' \J{->} \tau_r'
    %   \implies
    %   \delta \satisfies \tau_l \subtypes \tau_l'
    %   \implies
    %   \delta \satisfies \tau_r' \subtypes \tau_r
    % } {
    %   \delta \satisfies f\ \J{[}p \J{=>} e \J{]}\hastype \tau_l \J{->} \tau_r
    % }

    % \inferrule { 
    %   \delta \satisfies f \hastype \tau_l \J{->} \tau_r
    %   \\\\
    %   \forall \tau_l', \tau_r' \qua
    %   \delta \satisfies \J{[}p \J{=>} e \J{]} \hastype \tau_l' \J{->} \tau_r'
    %   \implies
    %   \delta \satisfies \tau_l \subtypes \tau_l'
    %   \implies
    %   \delta \satisfies \tau_r' \subtypes \tau_r
    % } {
    %   \delta \satisfies \J{[}p \J{=>} e \J{]}\ f \hastype \tau_l \J{->} \tau_r
    % } 

    % \inferrule { 
    %   \TODO{how to deal with recursive negative occurence?}
    %   \\\\
    %   \forall v \qua 
    %   \delta \satisfies v \hastype \tau_l
    %   \implies
    %   \exists \sigma \qua
    %   v \simeq p \given \sigma 
    %   \up
    %   \delta  \satisfies e[\sigma] \hastype \tau_r
    % } {
    %   \delta \satisfies \J{[}p \J{=>} e \J{]}\ f\hastype \tau_l \J{->} \tau_r
    % }

    % \inferrule { 
    %   \TODO{how to deal with recursive negative occurence?}
    %   \\\\
    %   \TODO{intersection elim is built into pattern matching}
    %   \\\\
    %   \forall v \qua 
    %   \delta \VDash v \hastype \rho
    %   \implies
    %   \exists \sigma \qua
    %   v \simeq p \given \sigma 
    %   \up
    %   \delta  \satisfies e[\sigma] \hastype \tau_r
    % } {
    %   \delta \satisfies \J{[}p \J{=>} e \J{]}\ f\hastype \rho \J{->} \tau_r
    % }

  \end{mathpar}
\end{definition}

\noindent
Model typing describes what expressions inhabit what types,
according to the forms of types, the forms of values, and how expressions
progress into values. 
It imbues types with semantics, which we use
to guide and judge the design of our type inference system. 
In general, a typing holds if its expression can make progress and 
the typing holds for the reduced form.

For inhabitation of a entry \ms{\delta \satisfies g \hastype \J{<}l\J{>}\tau},
the typing holds if the the expression is a record value, and
there is an entry in the record whose content inhabits     
the content of the entry type. Additionally, there must be just one
value per entry in the record. 
For inhabitation of a path 
\ms{\delta \satisfies f\ \J{[}p \J{=>} e \J{]} \hastype \tau_l\J{->}\tau_r},
the typing holds if the the expression is a function,
and there is one path in the function where 
the typing of the path body holds whenever the typing of the pattern holds.   
Additionally, since the paths of a function are tried in order, such that 
only the result of the first matched path is returned,
it is required that either the antecedent of the path type
is not weaker than the antecedent of preceding paths 
or the consequent of the path type is weaker than the consequent
of preceding paths. 

For inhabitation of a difference 
\ms{\delta \satisfies e \hastype \tau_l \typdiff \tau_r},
the typing holds if the expression inhabits the left type
but not the right right.
For inhabitation of a intersection 
\ms{\delta \satisfies e \hastype \tau_l \J{\&} \tau_r},
the typing holds if the expression inhabits both the
left type and the right type. 
For inhabitation of a union 
\ms{\delta \satisfies e \hastype \tau_l \J{|} \tau_r},
the typing holds if the expression inhabits either the
left type or the right type. 

For inhabitation of a universal 
\ms{\delta \satisfies e \hastype \J{ALL[} \Theta \J{]}\Delta\J{:}\tau},
the typing holds if the expression inhabits the constituent type,
under all interpretations of bound variables that satisfy the qualifiers. 
To avoid accepting an erroneous expression inhabiting a vacuous
type, it is required that the qualifiers are satisfiable.  
For inhabitation of an existential 
\ms{\delta \satisfies e \hastype \J{ALL[} \Theta \J{]}\Delta\J{:}\tau},
the typing holds if qualifiers and the inhabitation of the constituent type
are satisfiable by an interpretation of the bound variables.

For inhabitation of a variable \ms{\delta \satisfies e \hastype \alpha},
the typing holds if there's an interpretation for which the typing holds. 
For inhabitation of unit \ms{\delta \satisfies \J{@} \hastype \J{@}},
the typing holds if the expression is unit.
For inhabitation of a least fixed point \ms{\delta \satisfies e \hastype \J{LFP[}\alpha\J{]}\alpha},
the typing holds if the expression also inhabits the inflated type. 

\begin{definition}
  \label{def:model_subtyping}
  \emph{model subtyping}
  \hfill
  \small
  \boxed{\delta \satisfies \tau <: \tau}
  \nopad
  \begin{mathpar}
    \inferrule { 
      \forall e .\ 
      \delta \satisfies e \hastype \tau_l \implies 
      \delta \satisfies e \hastype \tau_r 
    } {
      \delta \satisfies \tau_l <: \tau_r
    } 
  \end{mathpar}
\end{definition}

\noindent
Model subtyping describes how the spaces of two types compare to each other.

\begin{definition}
  \label{def:model_subtyping_sequence}
  \emph{model subtyping sequence} 
  \hfill
  \small
  \boxed{\delta \satisfies \Delta}
  \nopad
  \begin{mathpar}
    \inferrule { 
    } {
      \delta \satisfies \epsilon 
    } 

    \inferrule { 
      \delta \satisfies \Delta 
      \\
      \delta \satisfies \tau_l \subtypes \tau_r
    } {
      \delta \satisfies \Delta \  \tau_l \J{<:} \tau_r
    } 
  \end{mathpar}
\end{definition}

\noindent
Model subtyping sequence describes multiple subtyping constraints. 

\begin{definition}
  \label{def:model_typing_sequence}
  \emph{model typing sequence} 
  \hfill
  \small
  \boxed{\delta, \sigma \satisfies \Gamma}
  \nopad
  \begin{mathpar}
    \inferrule {
      \forall x \qua
      x \hastype \tau \in \Gamma
      \implies
      \delta \satisfies x[\sigma] \hastype \tau
    } {
      \delta, \sigma \satisfies \Gamma
    }
  \end{mathpar}
\end{definition}

\noindent
Model typing sequence describes multiples typings of variables.

\section{Static System}
\label{sec:static_system}

The static system consists of ways to characterize abstract expressions 
and abstract types without relying on progression to concretize and reduce expressions. 
The two main components are proof typing and proof subtyping.
Proof typing is a strengthening of model typing, and 
proof subtyping is a strengthening of model subtyping.
When viewed as an algorithm,
they form the core of the type inference system. 
Proof typing generates subtyping constraints approximating the control flow of an expression, 
and then leverages proof subtyping to solve for the variables in the the subtyping constraints,
where the variables represent possible types of expressions.


\begin{definition} 
  \label{def:proof_typing}
  \emph{proof typing}
  \hfill 
  \small
  \boxed{\Theta, \Delta, \Gamma \entails e \hastype \tau \given \Theta', \Delta'}
  \nopad
  \begin{mathpar}
    \inferrule {
    } {
      \Theta, \Delta, \Gamma \entails \J{@} \hastype \J{@} \given \Theta, \Delta
    }

    \inferrule {
      x:\tau \in \Gamma
    } {
      \Theta, \Delta, \Gamma \entails x \hastype \tau \given \Theta, \Delta 
    }

    \inferrule {
    } {
      \Theta, \Delta, \Gamma \entails \epsilon \hastype \J{TOP} \given \Theta, \Delta
    }

    \inferrule {
      \Theta, \Delta, \Gamma \entails r \hastype \tau \given \Theta', \Delta'  
      \\\\
      \Theta', \Delta', \Gamma \entails e \hastype \tau' \given \Theta'', \Delta'' 
    } {
      \Theta, \Delta, \Gamma \entails r\ \J{<}l \J{>} e \hastype (\tau\ \J{\&}\ \J{<} l \J{>} \tau') 
      \given \Theta'', \Delta'' 
    }

    \inferrule {
      \exists \Xi \qua \Theta, \Delta, \Gamma \entails f \liftfun \Pi \sim \Xi 
      \\\\
      \textbf{pack}^+(\Theta \cup \textbf{ftv}(\Delta) \cup \textbf{ftv}(\Gamma) \entails \Pi) = \tau 
    } {
      \Theta, \Delta, \Gamma \entails f \hastype \tau \given \Theta, \Delta
    }

    \inferrule {
      \Theta, \Delta, \Gamma \entails e \hastype \tau \given \Theta', \Delta' 
      \\\\
      \Theta', \Delta' \entails 
      \tau \subtypes  \J{<} l \J{>} \alpha 
      \given \Theta'', \Delta'' 
    } {
      \Theta, \Delta, \Gamma \entails e\J{.}l \hastype \alpha \given \Theta'', \Delta''
    }

    \inferrule {
      \Theta, \Delta, \Gamma \entails e_f \hastype \tau_f \given \Theta', \Delta'
      \\
      \Theta', \Delta', \Gamma \entails e_a \hastype \tau_a \given \Theta'', \Delta''
      \\\\
      \Theta'', \Delta'', \tau_f \subtypes \tau_a\J{->}\alpha \given \Theta''', \Delta'''
    } {
      \Theta, \Delta, \Gamma \entails e_f\J{(}e_a \J{)} \hastype \alpha \given \Theta''', \Delta'''
    }

    \inferrule {
      \textbf{ftv}(\tau_a) \subseteq \emptyset
      \\
      \Theta, \Delta, \Gamma \entails e \hastype \Pi
      \\
      \textbf{pack}^+(\Theta \cup \textbf{ftv}(\Delta) \cup \textbf{ftv}(\Gamma) \entails \Pi) = \tau 
      \\\\
      \exists \Theta^\dagger, \Delta^\dagger \qua \Theta', \Delta' \entails \tau \subtypes \tau_a \given \Theta^\dagger, \Delta^\dagger
      \\
      \Theta', \Delta', \Gamma, x:\tau_a \entails e' \hastype \tau' \given \Theta'', \Delta''
    } {
      \Theta, \Delta, \Gamma \entails \J{def } x \J{:} \tau_a \J{ = } e \J{ in } e' \hastype \tau' \given \Theta'', \Delta''
    }

    \inferrule {
      \TODO{modify with case for non-recursive parameter type}
      \\\\
      \Theta, \Delta, \Gamma \entails e \hastype \tau \given  \Theta', \Delta'
      \\
      \Theta', \Delta' \entails \tau \subtypes \alpha \rightarrow \Pi
      \\
      \alpha \notin \textbf{ftv}(\Delta')
      \\\\
      \alpha \downarrow \Pi
      \fallingdotseq 
      \alpha' \uparrow \Pi'
      \\
      \textbf{pack}^-(\textbf{ftv}(\Delta')\ \alpha' \entails \Pi') = \tau'
      \\\\
      \factorsinto \J{LFP[} \alpha' \J{]}\tau' \subtypes \J{left} \mapsto \tau_l
      \\
      \factorsinto \J{LFP[} \alpha' \J{]}\tau' \subtypes \J{right} \mapsto \tau_r
    } {
      \Theta, \Delta, \Gamma \entails \J{loop(}e\J{)} 
      \hastype 
      \tau_l \J{->} \tau_r
      \given \Theta', \Delta'
    }
  \end{mathpar}
\end{definition}

\noindent
For a given context and expression, 
proof typing attempts to infer a type and a subtyping sequence,
such that if the context holds,
then there is a solution to the generated subtyping sequence, under which 
the expression inhabits the inferred type.
The context consists of skolems \ms{\Theta}, subtyping constraints \ms{\Delta}, and typings \ms{\Gamma}. 
The contextual subtyping constraints and typings constrain the search space of valid typings, 
such that the generated subtyping constraints are consistent with them. 
In fact, the generated subtyping constraints contain
all the contextual subtyping constraints, as it keeps the definition a bit less cluttered.
Skolems are variables that further constrain the search space. 
The generated subtyping constraints must be satisfiable under every interpretation of
skolems for which the contextual constraints are satisfiable.  
Proof typing may also generate additional skolems, which can be used to constrain
subsequent sub-problems.
If a variable is not in the set of skolems, then we consider it to be placeholder,
in that it can be constrained in any way that is consistent with the other subtyping constraints
and the typing. 



\noindent
For unit \code{@}, it infers the unit type \J{@}. 
For a variable, \ms{x}, it looks for a corresponding typing in the context, 
and infers the type it finds.
For an empty sequence \ms{\epsilon}, 
it infers top \code{TOP},
thereby providing a basis for inferring types of records. 
For a record \ms{r\ \J{<}l\J{>}e},  
it recursively constructs the type for each entry, and
refines their types against each other by compounding intersection.

For a function \ms{f},  
it delegates the work to two helpers. 
First, it leverages function typing consolidation 
\ms{\Theta, \Delta, \Gamma \entails f \liftfun \Pi \sim \Xi}
to infer as many types as possible
and consolidates the results into a sequence of locales \ms{\Pi}.
It must consolidate all of the inferred types in order to
ensure that the inferred type of each 
path's pattern does not subtype the type of subsequent paths' patterns. 
The sequence of subtractable types \ms{\Xi} can be ignored, as it is only
exposed for the sake of being used recursively.
Next, it leverage positive packing 
\ms{\textbf{pack}^+(... \entails \Pi) = \tau}
to rewrite each locale into a type,
such that the skolems and placeholders are converted to
existential and universal bindings,
and then combine all the types by compounding intersection.
Packing the path types together ensures that certain
control flows of application can be verified even when  
the best approximation of the type of the argument
is less precise than the pattern of any one path.


For projection \ms{e\J{.}l}, 
it delegates to proof subtyping 
\ms{... \entails \tau \subtypes \J{<}l\J{>}\alpha \given ...}
to check that there is an entry with the label \ms{l}
in the source expression's type \ms{\tau}.
With subtyping, it learns a lower bound on the type of the entry's content \ms{\alpha}.
For application \ms{e_f\J{(}e_a\J{)}}, 
it delates to proof subtyping 
\ms{... \entails \tau_f \subtypes \tau_a \J{->} \alpha \given ...}
to check that that the function \ms{e_f} can actually
map the argument \ms{e_a} to a result. With subtyping, 
it learns a lower bound on the type of the result \ms{\alpha}.
For an annotation \ms{\J{def}\ x \J{:} \tau_a\ \J{=}\ e\ \J{in}\ e'}, 
it first delegates to a few helpers. 
First it leverages typing consolidation \ms{\Theta, \Delta, \Gamma \entails e \hastype \Pi}
to infer as many possible locales for the source expression \ms{e}.
Then it packs all the locales together and ensures that the packed type
subtypes the type annotation.
It assigns the annotated type to the defined variable in the contextual typing
as it proceeds to infer the type of the continuation \ms{e'}, thereby
artificially restricting the permissible control flow. 


For a loop \ms{\J{loop(}e\J{)}}, 
it eagerly factors the paths of the recursive function 
to construct a single path type 
with an recursive type as its antecedent and an recursive type as its consequent. 
First, it infers the type of the fixed point combinators argument \ms{e},
and it leverages loop subtyping consolidation 
\ms{\Theta', \Delta' \entails \tau \subtypes \alpha_\nu \rightarrow \Pi^\nu}
to ensure the argument has a  
form compatible with looping and 
to collect as many type locales of its paths as possible.  
It leverage locale sequence duality 
\ms{
  \alpha_\nu \downarrow \Pi_\nu
  \fallingdotseq 
  \alpha_\mu \uparrow \Pi_\mu
}, which rewrites the implicitly co-recursive path locales into
implicitly recursive pair locales. 
Then, it leverages negative packing \ms{\textbf{pack}^-(... \entails \Pi_\mu) = \tau_\mu},
and constructs an explicitly recursive type by wrapping it in a least fixed point
\ms{\J{LFP[}\alpha_\mu\J{]}\tau_\mu}.
It then leverages factorization \ms{
  \Vdash \J{LFP[} \alpha_\mu \J{]}\tau_\mu \subtypes \J{<left>}\ \tau_l
} to factor out the left part of the recursive pair, and does the same for the right part. 
Finally from the left and right parts, it forms a single path type
with a least fixed point for both the antecedent and the consequent. 


\begin{definition} 
  \label{def:proof_subtyping}
  \emph{proof subtyping}
  \hfill
  \small
  \boxed{\Theta, \Delta \entails \tau_l \subtypes \tau_r \given \Theta', \Delta'}
\end{definition}

\noindent
For a given context and two abstract types,
proof subtyping attempts to solve the subtyping between the two types,
by generating a subtyping sequence, such that if the context holds,
then there is a solution to the generated subtyping sequence, under which   
the initial subtyping also holds. 

The context consists of skolems \ms{\Theta} and subtyping constraints \ms{\Delta}. 
The contextual subtyping constraints constrain the search space of valid subtyping constraints, 
such that the generated subtyping constraints are consistent with them. 
In fact, the generated subtyping constraints contain all the contextual subtyping constraints.
As with proof typing, skolems constrain the search space,
such that the generated subtyping constraints must be satisfiable
for every interpretation of skolems for which the contextual subtyping constraints are satisfiable.

In order to find a solution, the solver considers the forms 
of both types in the subtyping relation.
There are elimination rules, which reason according to the form of the subtype,
and there are introduction rules, which reason according to the form of the supertype.
Some rules handle both elimination and introduction of type forms.
Due to the numerous forms of types where each form is considered for each of the two types,
proof subtyping relies on many rules.  
For clarity, we subdivide into the rules into smaller groups according to some themes.

\begin{definition} 
  \label{def:proof_subtyping_reflection}
  \emph{proof subtyping (reflection)}
  \hfill
  \small
  \boxed{\Theta, \Delta \entails \tau_l \subtypes \tau_r \given \Theta', \Delta'}
  \nopad
  \begin{mathpar}
    \inferrule {
    } {
      \Theta, \Delta \entails \tau \subtypes \tau \given \Theta, \Delta 
    }
  \end{mathpar}
\end{definition}

\noindent
Proof subtyping is reflexive. A type
is considered both weaker and stronger than itself. 
We implicitly assume alpha conversion for variables
bound in existentials and universals.
Our implementation of proof typing converts to a nameless representation,
based on de Bruijn indexes \cite{} before checking equality.

\begin{definition} 
  \label{def:proof_subtyping_implication_preservation}
  \emph{proof subtyping (implication preservation)}
  \hfill
  \small
  \boxed{\Theta, \Delta \entails \kappa_l \subtypes \kappa_r \given \Theta', \Delta'}
  \nopad
  \begin{mathpar}
    \inferrule {
      \Theta, \Delta \entails 
      \tau_l \subtypes \tau_r
      \given \Theta', \Delta' 
    } {
      \Theta, \Delta \entails 
      \J{<}l \J{>} \tau_l \subtypes \J{<} l \J{>} \tau_r
      \given \Theta', \Delta'
    }

    \inferrule {
      \Theta, \Delta \entails \tau_{x} \subtypes \tau_{p} \given \Theta', \Delta'
      \\
      \Theta', \Delta' \entails \tau_{q} \subtypes \tau_{y} \given \Theta'', \Delta'' 
    } {
      \Theta, \Delta \entails 
      \tau_{p} \J{->} \tau_{q} 
      \subtypes 
      \tau_{x} \J{->} \tau_{y}
      \given \Theta'', \Delta'' 
    }
  \end{mathpar}
\end{definition}

\noindent
Both entry types and path types are analogous to propositional implication. 
However, unlike the typical separate rules for introduction and elimination in propositional implication,
the subtyping solver combines the semantics for introduction
and elimination into rules that preserve the implication structures.

For preservation of entries 
\ms{\J{<}l \J{>} \tau_l \subtypes \J{<} l \J{>} \tau_r},
it checks that the entries' labels match and that 
subtyping holds for their contents.
For preservation of paths 
\ms{\tau_p \J{->} \tau_q \subtypes \tau_x \J{->} \tau_y},
it checks that the antecedent of the subtype
is weaker than the antecedent of the supertype, 
and the consequent of the subtype 
is stronger than the consequent of the supertype. 

\begin{definition} 
  \label{def:proof_subtyping_expansion_elimination}
  \emph{proof subtyping (expansion elimination)}
  \hfill
  \small
  \boxed{\Theta, \Delta \entails \phi \subtypes \tau \given \Theta', \Delta'}
  \nopad
  \begin{mathpar}
    \inferrule {
      \Theta, \Delta \entails
      \tau_a \subtypes \tau
      \given \Theta', \Delta'
      \\
      \Theta', \Delta' \entails
      \tau_b \subtypes \tau
      \given \Theta'', \Delta''
    } {
      \Theta, \Delta \entails
      \tau_a\J{|}\tau_b \subtypes \tau
      \given \Theta'', \Delta'' 
    }

    \inferrule {
      \Theta, \Delta \Vvdash \Omega
      \\
      \Theta, \Delta \entails \Omega \given \Theta', \Delta'
      \\
      \Theta' \cup \Theta_l, \Delta' \entails
      \tau_l \subtypes \tau_r \given \Theta'', \Delta''
    } {
      \Theta, \Delta \entails
      \J{EXI[}\Theta_l\J{]}\Omega \J{:} \tau_l \subtypes \tau_r
      \given \Theta'', \Delta'' 
    }
  \end{mathpar}
\end{definition}

\noindent
For union elimination \ms{
\tau_a\J{|}\tau_b \subtypes \tau
},
it solves elimination for both of union's constituents.
For existential elimination \ms{ 
\J{EXI[}\Theta_l \J{]} \Omega \J{:}\tau_l \subtypes \tau_r
}, 
it ensures that the existential's qualifiers
are negatable \ms{\Theta,\Delta \Vvdash \Omega}, 
and then it finds a solution for the qualifiers.
It then solves the elimination of the existential's constituent type,
constrained by the skolemized bound variables,
and the solution to the qualifiers. 
The negatable check is necessary to ensure
that a solution the the qualifiers implies
a solution to the subtyping constraints generated by solving the qualifiers.
Thus, if the solver can eliminate the existential, 
then a solution to the qualifiers implies a solution to the constituent's elimination.


\begin{definition} 
  \label{def:proof_subtyping_refinement_introduction}
  \emph{proof subtyping (refinement introduction)}
  \hfill
  \small
  \boxed{\Theta, \Delta \entails \tau \subtypes \psi \given \Theta', \Delta'}
  \nopad
  \begin{mathpar}
    \inferrule {
      \Theta, \Delta \entails
      \tau \subtypes \tau_{a} \given \Theta', \Delta' 
      \\
      \Theta', \Delta' \entails
      \tau \subtypes \tau_{b} \given \Theta'', \Delta''
    } {
      \Theta, \Delta \entails
      \tau \subtypes \tau_{a}\J{\&}\tau_{b} \given \Theta'', \Delta''
    }

    \inferrule {
      \Theta, \Delta \Vvdash \Omega
      \\
      \Theta, \Delta \given \Omega \given \Theta', \Delta'
      \\
      \Theta' \cup \Theta_r, \Delta' \entails \tau_l \subtypes \tau_r \given \Theta'', \Delta'' 
    } {
      \Theta, \Delta \entails \tau_l \subtypes \J{ALL[}\Theta_r\J{]}\Omega \J{:} \tau_r \given \Theta'', \Delta''
    }
  \end{mathpar}
\end{definition}

\noindent
Refinement introduction is the dual of expansion elimination, so its rules
rely on analogous techniques.

\begin{definition} 
  \label{def:proof_subtyping_placeholder_elimination}
  \emph{proof subtyping (placeholder elimination)}
  \hfill
  \small
  \boxed{\Theta, \Delta \entails \alpha \subtypes \tau \given \Theta', \Delta'} 
  \nopad
  \begin{mathpar}
    \inferrule {
      \alpha \notin \Theta
      \\
      (\forall \tau' \qua \tau' \J{<:} \alpha \in \Delta \implies \tau' \J{<:} \tau \in \Delta^\dagger)
      \\
      \Theta, \Delta \entails \Delta^\dagger \given \Theta', \Delta'
    } {
      \Theta, \Delta \entails \alpha \subtypes \tau \given \Theta', \Delta'\ \alpha\J{<:}\tau
    }
    % \inferrule {
    %   \alpha \notin \Theta
    %   \\
    %   \Theta, \Delta \entails \Delta[\alpha\slash\tau]^{+} \backslash \Delta \given \Theta', \Delta'
    % } {
    %   \Theta, \Delta \entails \alpha \subtypes \tau \given \Theta', \Delta'\ \alpha\J{<:}\tau
    % }
  \end{mathpar}
\end{definition}

\noindent
For placeholder elimination \ms{\alpha \subtypes \tau} where \ms{\alpha \notin \Theta}, 
it ensures that the supertype is consistent with all lower bounds of the placeholder,
solutions against those lower bounds. Then it learns that the supertype is an upper bound 
of the placeholder.

\begin{definition} 
  \label{def:proof_subtyping_placeholder_introduction}
  \emph{proof subtyping (placeholder introduction)}
  \hfill
  \small
  \boxed{\Theta, \Delta \entails \tau \subtypes \alpha \given \Theta', \Delta'}
  \nopad
  \begin{mathpar}
    \inferrule {
      \alpha \notin \Theta
      \\
      (\forall \tau' \qua \alpha \J{<:} \tau' \in \Delta \implies \tau \J{<:} \tau' \in \Delta^\dagger) 
      \\
      \Theta, \Delta \entails \Delta^\dagger \given \Theta', \Delta'
    } {
      \Theta, \Delta \entails 
      \tau \subtypes \alpha \given \Theta', \Delta'\ \tau\J{<:}\alpha
    }
    % \inferrule {
    %   \alpha \notin \Theta
    %   \\
    %   \Theta, \Delta \entails \Delta[\alpha\slash\tau]^{-} \backslash \Delta \given \Theta', \Delta'
    % } {
    %   \Theta, \Delta \entails 
    %   \tau \subtypes \alpha \given \Theta', \Delta'\ \tau\J{<:}\alpha
    % }
  \end{mathpar}
\end{definition}

\noindent
Placeholder introduction is the dual of placeholder elimination, so its rule
relies on analogous techniques.

\begin{definition} 
  \label{def:proof_subtyping_skolem_introduction}
  \emph{proof subtyping (skolem introduction)}
  \hfill
  \small
  \boxed{\Theta, \Delta \entails \tau \subtypes \alpha \given \Theta', \Delta'}
  \nopad
  \begin{mathpar}
    \inferrule {
      \alpha \in \Theta 
      \\
      \exists \alpha' \qua \alpha' \J{<:} \alpha \in \Delta \up \alpha' \notin \Theta 
      % \\\\
      % \exists \Theta', \Delta' \qua \Theta, \Delta \entails \Delta[\alpha\slash\tau]^{-} \backslash \Delta \given \Theta', \Delta'
      \\\\
      (\forall \tau' \qua \alpha \J{<:} \tau' \in \Delta \implies \tau \J{<:} \tau' \in \Delta^\dagger)
      \\
      \Theta, \Delta \entails 
      \Delta^\dagger 
      \given \Theta', \Delta'
    } {
      \Theta, \Delta \entails 
      \tau \subtypes \alpha 
      \given \Theta', \Delta'\ \tau \J{<:} \alpha 
    }

    \inferrule {
      \alpha \in \Theta 
      \\
      \tau' \J{<:} \alpha \in \Delta
      \\\\
      (\forall \alpha' \qua \tau' = \alpha' \implies \alpha' \in \Theta) 
      \\
      \Theta, \Delta \entails \tau \subtypes \tau' \given \Theta', \Delta'
    } {
      \Theta, \Delta \entails 
      \tau \subtypes \alpha 
      \given \Theta', \Delta' 
    }
  \end{mathpar}
\end{definition}

\noindent
For skolem introduction \ms{\tau \subtypes \alpha} where \ms{\alpha \in \Theta},
there are two possibilities.
In the first scenario, the skolem has a placeholder as a lower bound.
If effect, this enables learning lower bounds on the skolem,
so the solver uses the same techniques as it did for placeholder elimination.
I the second scenario, there is a lower bound that's not a placeholder,
so the solver ensures that the skolem introduction is weaker
than the constraints by checking that the subtype subtypes the lower bound that
was found. In practice, the solver avoids checking the subtype against the lower bound, 
if the lower bound is a placeholder, as that would cause the solver to diverge due to
a mutual dependence between placeholder elimination and skolem elimination.      

\begin{definition} 
  \label{def:proof_subtyping_skolem_elimination}
  \emph{proof subtyping (skolem elimination)}
  \hfill
  \small
  \boxed{\Theta, \Delta \entails \alpha \subtypes \tau \given \Theta', \Delta'} 
  \nopad
  \begin{mathpar}
    \inferrule {
      \alpha \in \Theta 
      \\
      \exists \alpha' \qua \alpha \J{<:} \alpha' \in \Delta \up \alpha' \notin \Theta 
      % \\\\
      % \exists \Theta', \Delta' \qua \Theta, \Delta \entails \Delta[\alpha\slash\tau]^{+} \backslash \Delta \given \Theta', \Delta'
      \\\\
      (\forall \tau' \qua \tau' \J{<:} \alpha \in \Delta \implies \tau' \J{<:} \tau \in \Delta^\dagger)
      \\
      \Theta, \Delta \entails 
      \Delta^\dagger 
      \given \Theta', \Delta'
    } {
      \Theta, \Delta \entails 
      \alpha \subtypes \tau
      \given \Theta', \Delta'\ \alpha \J{<:} \tau
    }

    \inferrule {
      \alpha \in \Theta 
      \\
      \alpha \J{<:} \tau' \in \Delta
      \\\\
      (\forall \alpha' \qua \tau' = \alpha' \implies \alpha' \in \Theta) 
      \\
      \Theta, \Delta \entails \tau' \subtypes \tau \given \Theta', \Delta'
    } {
      \Theta, \Delta \entails 
      \alpha \subtypes \tau
      \given \Theta', \Delta'
    }
  \end{mathpar}
\end{definition}

\noindent
Skolem elimination is the dual of skolem introduction, so its rules
rely on analogous techniques.

\begin{definition} 
  \label{def:proof_subtyping_implication_rewriting}
  \emph{proof subtyping (implication rewriting)}
  \hfill
  \small
  \boxed{\Theta, \Delta \entails \tau_l \subtypes \tau_r \given \Theta', \Delta'}
  \nopad
  \begin{mathpar}
    \inferrule {
      \Theta, \Delta \entails 
      \tau_l \subtypes
      (\tau_a\J{->}\tau_r)\J{\&}(\tau_b\J{->}\tau_r)
      \given \Theta', \Delta'
    } {
      \Theta, \Delta \entails 
      \tau_l \subtypes 
      (\tau_a\J{|}\tau_b)\J{->}\tau_r
      \given \Theta', \Delta' 
    }


    \inferrule {
      \Theta, \Delta \entails 
      \tau_l
      \subtypes
      (\tau_r\J{->}\tau_a)
      \J{\&}
      (\tau_r\J{->}\tau_b)
      \given \Theta', \Delta' 
    } {
      \Theta, \Delta \entails 
      \tau_l \subtypes 
      \tau_r\J{->}(\tau_{a}\J{\&}\tau_{b})
      \given \Theta', \Delta'
    }

    \inferrule {
      \Theta, \Delta \entails 
      \tau
      \subtypes
      (\J{<} l \J{>} \tau_a)
      \J{\&}
      (\J{<} l \J{>} \tau_b)
      \given \Theta', \Delta' 
    } {
      \Theta, \Delta \entails 
      \tau \subtypes 
      \J{<} l \J{>} (\tau_{a}\J{\&}\tau_{b})
      \given \Theta', \Delta'
    }

  \end{mathpar}
\end{definition}

\noindent
Just as entries and paths are analogous to propositional implication,
unions and intersections are analogous to disjunction and conjunction, respectively.
Just as disjunction elimination or propositional logic relies on two cases of implication,
the solver recognizes that union in the antecedent of a path,
can be solved by rewriting into two cases of paths.  
Additionally, since must preserve implications, there is no direct no single
rule that corresponds modus ponens. Thus, the solver 
recognizes that intersection in the consequent of a path or (the content of an entry),
may need to be rewritten as two paths (or two entries).


\begin{definition} 
  \label{def:proof_subtyping_lfp_elimination}
  \emph{proof subtyping (least fixed point elimination)}
  \hfill
  \small
  \boxed{\Theta, \Delta \entails \J{LFP[}\alpha\J{]}\tau_l \subtypes \tau_r \given \Theta', \Delta'}
  \nopad
  \begin{mathpar}
    \inferrule {
      \Vdash \J{LFP[}\alpha\J{]}\tau_l \subtypes l \mapsto \tau_r' 
      \\
      \Theta, \Delta \entails \tau_r' \subtypes \tau_r \given \Theta', \Delta' 
    } {
      \Theta, \Delta \entails 
      \J{LFP[}\alpha\J{]}\tau_l \subtypes \J{<} l \J{>} \tau_r 
      \given \Theta', \Delta' 
    }

    \inferrule {
      \alpha \notin \textbf{ftv}(\tau_l)
      \\
      \Theta, \Delta \entails 
      \tau_l \subtypes \tau_r 
      \given \Theta', \Delta' 
    } {
      \Theta, \Delta \entails 
      \J{LFP[}\alpha\J{]}\tau_l \subtypes \tau_r 
      \given \Theta', \Delta' 
    }

    \inferrule {
      \tau_l[\alpha \slash \tau_r]^+ = \tau^\dagger
      \\
      \Theta, \Delta \entails 
      \tau^\dagger \subtypes \tau_r \given 
      \Theta', \Delta'
      \\
      \alpha \not\in \textbf{ftv}(\tau^\dagger)
      \\
      \alpha \not\in \textbf{ftv}(\tau_r)
    } {
      \Theta, \Delta \entails 
      \J{LFP[}\alpha\J{]}\tau_l \subtypes \tau_r 
      \given \Theta', \Delta' 
    }
  \end{mathpar}
\end{definition}

\noindent
For least fixed point elimination \ms{
\J{LFP[} \alpha \J{]} \tau_l \subtypes \tau_r
}, there are a few scenarios
that may work. 
In the first scenario, the solver recognizes that the supertype
is an entry, so it factors out the corresponding column from
the least fixed point.
In the second scenario, the solver recognizes that 
the constituent type doesn't contain the self reference,
so it's sufficient to simply solve the elimination of the constituent.  
In the third case, the solver performs mathematical induction.
It substitutes the super type \ms{\tau_r}
into all positive occurrences of the self reference \ms{\alpha} within
the least fixed point's constituent type \ms{\tau_l}.
It then solves the elimination of this substitution into the super type.
In effect, solving the least fixed point elimination under the positive substitution is 
akin to deriving the least fixed point elimination under the assumption of the induction hypothesis,
\ms{\alpha \subtypes \tau_r}.

\begin{definition} 
  \label{def:proof_subtyping_difference_intro}
  \emph{proof subtyping (difference introduction)}
  \hfill
  \small
  \boxed{\Theta, \Delta \entails \tau_l \subtypes \tau_r \typdiff \eta \given \Theta', \Delta'}
  \nopad
  \begin{mathpar}
    \inferrule {
      \TODO{Update proof, deriving that lower does not intersect with subtraction}
      \\\\
      \Theta, \Delta \Vvdash \eta
      \\
      \nexists \Theta', \Delta' \qua  
      \Theta, \Delta \entails 
      \tau_l \subtypes \eta \given \Theta', \Delta'
      \\\\
      \Theta, \Delta \Vvdash \eta \subtypes  \tau_l
      \\
      \nexists \Theta', \Delta' \qua  
      \Theta, \Delta \entails 
      \eta \subtypes \tau_l \given \Theta', \Delta'
      \\\\
      \Theta, \Delta \entails 
      \tau_l \subtypes \tau_r \given \Theta', \Delta' 
    } {
      \Theta, \Delta \entails 
      \tau_l \subtypes \tau_r \typdiff \eta
      \given \Theta', \Delta'
    }
  \end{mathpar}
\end{definition}

\noindent
For difference introduction \ms{\tau \subtypes \tau_r \typdiff \eta}, 
it checks that subtyping holds for the positive part 
and fails for the subtracted part.
Additionally, it requires that the subtracted part is subtractable 
\ms{\Theta, \Delta \Vvdash \eta}, 
which ensures that its introduction is complete,
thus failure of its introduction is sound. 

\begin{definition} 
  \label{def:proof_subtyping_lfp_intro}
  \emph{proof subtyping (least fixed point introduction)}
  \hfill
  \small
  \boxed{\Theta, \Delta \entails \tau_l \subtypes \J{LFP[} \alpha \J{]}\tau_r \given \Theta', \Delta'}
  \nopad
  \begin{mathpar}
    \inferrule {
      \Theta, \Delta \entails \tau_l \circlearrowleft \J{LFP[} \alpha \J{]} \tau_r
      \\
      \Theta, \Delta \entails
      \tau_l \subtypes \tau_r[\alpha \slash \J{LFP[} \alpha \J{]} \tau_r]
      \given \Theta', \Delta'
    } {
      \Theta, \Delta \entails
      \tau_l \subtypes \J{LFP[} \alpha \J{]}\tau_r \given \Theta', \Delta'
    }

    \inferrule {
      \alpha \Vdash \tau_r \supertypes \tau_r'
      \\
      \Theta, \Delta \entails
      \tau_l \subtypes \tau_r' \given \Theta', \Delta'
    } {
      \Theta, \Delta \entails
      \tau_l \subtypes \J{LFP[} \alpha \J{]}\tau_r \given \Theta', \Delta'
    }

  \end{mathpar}
\end{definition}

\noindent
For least fixed point introduction \ms{
\tau_l \subtypes \J{LFP[} \alpha \J{]}\tau_r
}, the solver has a couple options.
One option is to check for a syntactic guarantee that the least fixed point is inflatable without diverging 
\ms{
\tau_l \circlearrowleft \J{LFP[} \alpha \J{]} \tau_r
} and then solve the inflated introduction.
Alternatively, it can strengthen the least fixed point
with union drop \ms{\alpha \Vdash \tau_r \supertypes \tau_r'},
which drops any parts of the union containing the self reference. 

\begin{definition} 
  \label{def:proof_subtyping_diff_elimination}
  \emph{proof subtyping (difference elimination)}
  \hfill
  \small
  \boxed{\Theta, \Delta \entails \tau_l \typdiff \eta \subtypes \tau_r \given \Theta', \Delta'}
  \nopad
  \begin{mathpar}
    \inferrule {
      \Theta, \Delta \entails 
      \tau_l \subtypes \eta \J{|} \tau_r \given \Theta', \Delta'
    } {
      \Theta, \Delta \entails 
      \tau_l \typdiff \eta \subtypes \tau_r \given \Theta', \Delta'
    }
  \end{mathpar}
\end{definition}

\noindent
For difference elimination \ms{ 
\rho \typdiff \eta \subtypes \tau
},
it simply rewrites the problem in terms of union.

\begin{definition} 
  \label{def:proof_subtyping_expansion_introduction}
  \emph{proof subtyping (expansion introduction)}
  \hfill
  \small
  \boxed{\Theta, \Delta \entails \tau \subtypes \phi \given \Theta', \Delta'}
  \nopad
  \begin{mathpar}
    \inferrule {
      \Theta, \Delta \entails \tau \subtypes \tau_{l} \given \Theta', \Delta' 
    } {
      \Theta, \Delta \entails \tau \subtypes \tau_{l}\J{|}\tau_{r} \given \Theta', \Delta' 
    }

    \inferrule {
      \Theta, \Delta \entails \tau \subtypes \tau_{r} \given \Theta', \Delta' 
    } {
      \Theta, \Delta \entails \tau \subtypes \tau_{l}\J{|}\tau_{r} \given \Theta', \Delta' 
    }

    \inferrule {
      \Theta, \Delta \entails \tau_l \subtypes \tau_r \given \Theta', \Delta'
      \\\\
      \Theta', \Delta' \entails \Delta_r \given \Theta'', \Delta'' 
    } {
      \Theta, \Delta \entails 
      \tau_l
      \subtypes 
      \J{EXI[}\Theta_r\J{]}\Delta_r \J{:} \tau_r \given \Theta'', \Delta'' 
    }
  \end{mathpar}
\end{definition}
\noindent
For union introduction \ms{
\tau \subtypes \tau_{l}\J{|}\tau_{r}
}, 
it can choose from two possible options:
solving the introduction of of left part, 
or solving the introduction of the right part.
For existential introduction \ms{ 
\tau_l \subtypes \J{EXI[}\Theta_r \J{]} \Delta_r \J{:}\tau_r
},
it constructs witnesses by solving the introduction
of the existential's constituent type \ms{\entails \tau_l \subtypes \tau_r \given}, 
from which the witnesses can be inferred from the generated subtyping constraints.
It then attempts to solve the qualifiers by leveraging the witnesses.

\begin{definition} 
  \label{def:proof_subtyping_refinement_elimination}
  \emph{proof subtyping (refinement elimination)}
  \hfill
  \small
  \boxed{\Theta, \Delta \entails \psi \subtypes \tau \given \Theta', \Delta'}
  \nopad
  \begin{mathpar}
    \inferrule {
      \Theta, \Delta \entails \tau_{l} \subtypes \tau \given \Theta', \Delta'
    } {
      \Theta, \Delta \entails \tau_{l}\J{\&}\tau_{r}  \subtypes \tau \given \Theta', \Delta' 
    }

    \inferrule {
      \Theta, \Delta \entails \tau_{l} \subtypes \tau \given \Theta', \Delta' 
    } {
      \Theta, \Delta \entails \tau_{l}\J{\&}\tau_{r}  \subtypes \tau \given \Theta', \Delta'
    }

    \inferrule {
      \Vdash \tau_{l}\J{\&}\tau_{r} \subtypes \tau_x \rightarrow \tau_y 
      \\\\
      \Theta, \Delta \entails \tau_x \J{->} \tau_y  \subtypes \tau_a \J{->} \tau_b \given \Theta', \Delta'
    } {
      \Theta, \Delta \entails \tau_{l}\J{\&}\tau_{r}  \subtypes \tau_a \J{->} \tau_b \given \Theta', \Delta'
    }

    \inferrule {
      \Theta, \Delta \entails \tau_l \subtypes \tau_r \given \Theta', \Delta'
      \\
      \Theta', \Delta' \entails \Delta_l \given \Theta'', \Delta'' 
    } {
      \Theta, \Delta \entails 
      \J{ALL[}\Theta_l\J{]} \Delta_l \J{:} \tau_l
      \subtypes 
      \tau_r
      \given \Theta'', \Delta'' 
    }
  \end{mathpar}
\end{definition}

\noindent
Refinement elimination is the dual of expansion introduction, so its rules
rely on analogous techniques.
However, there is one additional rule
for eliminating an intersection and introducing an implication \ms{
\tau_l \& \tau_r \subtypes \tau_a \J{->} \tau_b
}.
As we mentioned earlier when discussing packing together the inferred types of 
multiple paths of functions, sometimes the argument type is weaker than each
path type antecedent, yet could be stronger than union of all the path type antecedents.    
As such, the solver may leverage function type factorization \ms{
\Vdash \tau_l \J{\&} \tau_r \subtypes \tau_x \rightarrow \tau_y 
} to weaken a function type
into a single path type where the antecedent and the consequent may consist
of compounded unions. 

\begin{definition} 
  \label{def:proof_subtyping_sequence}
  \emph{proof subtyping sequence} 
  \hfill
  \small
  \boxed{\Theta, \Delta \entails \Delta_q \given \Theta', \Delta'}
  \nopad
  \begin{mathpar}
    \inferrule {
    } {
      \Theta, \Delta \entails \epsilon \given \Theta', \Delta'
    }

    \inferrule {
      \Theta, \Delta \entails \tau_l \J{<:} \tau_r \given \Theta', \Delta'
      \\
      \Theta', \Delta' \entails \tau_l \J{<:} \tau_r \given \Theta'', \Delta''
    } {
      \Theta, \Delta \entails \Delta_q\ \tau_l \J{<:} \tau_r \given \Theta'', \Delta''
    }
  \end{mathpar}
\end{definition}

\noindent
Proof subtyping sequence very simply solves a sequence of subtyping goals
and compounds their results, such that the solution from one constrains the next. 

\section{Theory}
\label{sec:theory}

Our design depends on numerous helpers to rewrite and check the syntax
of types. The details of these helpers play a key role in 
determining the level of expressivity allowed by the type inference system. 
However, these details are beyond the scope of the theory in work.
In this theory, we are agnostic to their specific details,
but we require that they meet certain requirements. 
From these abstract requirements, we can guarantee
that the design of our proof semantics is correct with respect
to our model semantics. 

\begin{requirement}
  \label{req:positive_substitution_soundness}
  \emph{positive substitution soundness}
  \small
  \nopad
  \begin{mathpar}
    \inferrule {
      \alpha \notin \textbf{ftv}(\tau[\alpha \slash \tau_r]^+)
    } {
      \delta \satisfies \tau_l \subtypes \tau_r
      \implies
      \delta\ \alpha \slash \tau_l \satisfies \tau \subtypes \tau[\alpha \slash \tau_r]^+
    }
  \end{mathpar}
\end{requirement}

\noindent
The requirement of positive substitution soundness says that
if a variable \ms{\alpha} only occurs in positive positions of a type \ms{\tau},
then the result of the substituting in another type \ms{\tau_r} is weaker than 
any interpretation of the original type \ms{\tau} 
if the replaced variable \ms{\alpha} is interpreted with a type \ms{\tau_l} 
that is weaker than the subbed in type \ms{\tau_r}.
This property is necessary so that our proof typing rules can use positive substitution
as a proxy reasoning under the induction hypothesis without violating soundness. 

\begin{requirement}
  \label{req:locale_sequence_duality_correspondence}
  \emph{locale sequence duality correspondence}
  \footnotesize
  \nopad
  \begin{mathpar}
    \inferrule {
      \alpha_\nu \downarrow \Pi_\nu \fallingdotseq \alpha_\mu \uparrow \Pi_\mu
      \\
      \delta \satisfies \Delta
      \\
      \alpha_\nu \notin \textbf{ftv}(\Delta)
      \\
      \alpha_\mu \notin \textbf{ftv}(\Delta)
    } {
      (
      \forall \Theta^\dagger, \Delta^\dagger, \tau^\dagger \qua
      (\Theta^\dagger,\Delta^\dagger, \tau^\dagger) \in \Pi_\nu \implies
      \exists \delta^\dagger \qua \textbf{dom}(\delta^\dagger) \subseteq \Theta^\dagger \up
      \\\\
      (
      \forall \delta' \qua
      \textbf{dom}(\delta') \cap \textbf{ftv}(\Delta) = \epsilon \implies
      % \delta\ \alpha_\nu \slash \J{BOT} \oplus \delta' \oplus \delta^\dagger \satisfies \Delta^\dagger \implies
      % \delta\ \alpha_\nu \slash \J{BOT} \oplus \delta' \oplus \delta^\dagger \satisfies e_f \hastype \tau^\dagger
      \delta \oplus \delta' \oplus \delta^\dagger \satisfies \Delta \cup \Delta^\dagger \implies
      \delta \oplus \delta' \oplus \delta^\dagger \satisfies e_f \hastype \tau^\dagger
      )
      )
      \\\\
      \iff
      \\\\
      (\forall e_a, e_r \qua
      (
      \exists \Theta^\dagger, \Delta^\dagger, \tau^\dagger \qua
      (\Theta^\dagger,\Delta^\dagger, \tau^\dagger) \in \Pi_\mu \up
      (\forall \delta^\dagger \qua \textbf{dom}(\delta^\dagger) \subseteq \Theta^\dagger \implies
      \\\\
      \exists \delta' \qua
      \textbf{dom}(\delta') \cap \textbf{ftv}(\Delta) = \epsilon \up
      % \delta\ \alpha_\mu \slash \J{TOP} \oplus \delta' \oplus \delta^\dagger \satisfies \Delta^\dagger \up
      % \delta\ \alpha_\mu \slash \J{TOP} \oplus \delta' \oplus \delta^\dagger \satisfies (e_a \J{,} e_r) \hastype \tau^\dagger 
      \delta \oplus \delta' \oplus \delta^\dagger \satisfies \Delta \cup \Delta^\dagger \up
      \delta \oplus \delta' \oplus \delta^\dagger \satisfies (e_a \J{,} e_r) \hastype \tau^\dagger 
      )
      )
      \implies
      e_r \cong e_f\J{(}e_a\J{)}
      )
    }
  \end{mathpar}
\end{requirement}


\noindent
The requirement of locale sequence duality has an ungodly amount of symbolic clutter,
but the essence of its meaning is actually quite simple and natural.
It says that a function inhabits all locales in some 
some collection if and only if for any input/ouput pair, if the pair inhabits some
locale from the dual collection, then the function applied to the input
gives the output. 
The requirement is essential to ensure that rewriting inferred locales
to construct least fixed point types does not violate soundness.



\begin{requirement}
  \label{req:positive packing correspondence}
  \emph{positive packing correspondence}
  \footnotesize
  \nopad
  \begin{mathpar}
    \inferrule {
      \textbf{pack}^+(\Theta \entails \Pi) = \tau 
      \\
      \delta \satisfies \Delta
      \\
      \Theta \subseteq \textbf{dom}(\delta)
    } {
      (\forall  \Theta^\dagger, \Delta^\dagger, \tau^\dagger \qua 
      (\Theta^\dagger, \Delta^\dagger, \tau^\dagger) \in \Pi \implies
      (
      \exists \delta^\dagger \qua \textbf{dom}(\delta^\dagger) \subseteq \Theta^\dagger \backslash \Theta \up 
      \\\\
      (
      \forall \delta' \qua 
      \textbf{dom}(\delta') \cap \textbf{ftv}(\Delta) = \emptyset \implies
      \delta \oplus \delta' \oplus \delta^\dagger \satisfies \Delta \cup \Delta^\dagger \implies
      \delta \oplus \delta' \oplus \delta^\dagger \satisfies e \hastype \tau^\dagger
      )
      ) 
      \\\\
      \iff
      \\\\
      \delta \satisfies e \hastype \tau
    }
  \end{mathpar}
\end{requirement}

\begin{requirement}
  \label{req:negative packing correspondence}
  \emph{negative packing correspondence}
  \footnotesize
  \nopad
  \begin{mathpar}
    \inferrule {
      \textbf{pack}^-(\Theta \entails \Pi) = \tau 
      \\
      \delta \satisfies \Delta
      \\
      \Theta \subseteq \textbf{dom}(\delta)
    } {
      (\exists  \Theta^\dagger, \Delta^\dagger, \tau^\dagger \qua 
      (\Theta^\dagger, \Delta^\dagger, \tau^\dagger) \in \Pi \up
      (
      \forall \delta^\dagger \qua \textbf{dom}(\delta^\dagger) \subseteq \Theta^\dagger \backslash \Theta \implies
      \\\\
      (
      \exists \delta' \qua 
      \textbf{dom}(\delta') \cap \textbf{ftv}(\Delta) = \emptyset \up
      \delta \oplus \delta' \oplus \delta^\dagger \satisfies \Delta \cup \Delta^\dagger \up
      \delta \oplus \delta' \oplus \delta^\dagger \satisfies e \hastype \tau^\dagger
      )
      ) 
      \\\\
      \iff
      \\\\
      \delta \satisfies e \hastype \tau
    }
  \end{mathpar}
\end{requirement}


\noindent


\begin{requirement}
  \label{req:proof_subtyping_sequence_completeness_under_negatability}
  \emph{proof subtyping sequence completeness under negatability} 
  \small
  \nopad
  \begin{mathpar}
    \inferrule {
      \Theta, \Delta \Vvdash \Omega
      \\
      \Theta, \Delta \entails \Omega \given \Theta', \Delta'
      \\
      \delta' \satisfies \Delta'
    } {
      \forall \delta^\dagger \qua
      \delta' \oplus \delta^\dagger \satisfies \Omega \implies
      \delta' \oplus \delta^\dagger \satisfies \Delta'
    }
  \end{mathpar}
\end{requirement}


\begin{requirement}
  \label{req:proof_subtyping_solution_completeness_under_negatability}
  \emph{proof subtyping solution completeness under negatability}  
  \small
  \nopad
  \begin{mathpar}
    \inferrule {
      \Theta, \Delta \Vvdash \eta
      \\
      \Theta, \Delta \entails \tau \subtypes \eta \given \Theta', \Delta'
      \\
      \delta' \satisfies \Delta'
    } {
      \forall \delta^\dagger \qua
      \delta' \oplus \delta^\dagger \satisfies \tau \subtypes \eta \implies
      \delta' \oplus \delta^\dagger \satisfies \Delta'
    }
  \end{mathpar}
\end{requirement}

\begin{requirement}
  \label{req:proof_subtyping_existential_completeness_under_negatability}
  \emph{proof subtyping existential completeness under negatability}  
  \small
  \begin{mathpar}
    \inferrule {
      \Theta, \Delta \Vvdash \eta
      \\
      \delta \satisfies \tau \subtypes \eta
      \\
      \delta \satisfies \Delta
    } {
      \exists \Theta', \Delta' \qua \Theta, \Delta \entails \tau \subtypes \eta \given \Theta', \Delta'
    }
  \end{mathpar}
\end{requirement}


\begin{requirement}
  \label{req:factorization_soundness}
  \emph{factorization soundness}  
  \small
  \begin{mathpar}
    \inferrule {
      \Vdash \J{LFP[}\alpha\J{]}\tau \subtypes l \mapsto \tau'
    } {
      \forall e' \qua
      e' \hastype \tau' 
      \iff 
      (\exists e \qua e \J{.} l \cong e' \up \delta \satisfies e \hastype \J{LFP[}\alpha\J{]}\tau)
    }
  \end{mathpar}
\end{requirement}




\begin{theorem}
  \label{thm:proof_subtyping_substance}
  \emph{proof subtyping substance}
  \small
  \nopad
  \begin{mathpar}
    \inferrule {
      \Theta, \Delta \entails \tau_l \subtypes \tau_r \given \Theta', \Delta'
    } {
      \forall \delta \qua \textbf{dom}(\delta) \subseteq \Theta \implies
      \exists \delta' \qua \delta \oplus \delta' \satisfies \Delta
      \implies
      \exists \delta' \qua \delta \oplus \delta' \satisfies \Delta'
    }
  \end{mathpar}
\end{theorem}

\noindent
Proof subtyping substance states that the generated constraints are satisfiable
assuming the contextual constraints are satisfiable, with respect
to all interpretations of the contextual skolems.   
This property follows directly from a slightly stronger theorem,
which states that the generated constraints are satisfiable and 
that the constraints on a single variable are as strong or stronger
than all the constraints with respect to that variable.
The proof of the stronger statement is by induction.
The only cases that are actually interesting are the variable rules
where new constraints are generated.
In every case that a new constraint is added, it is a constraint on
a single variable, and in each case the solver compares the new bound on the variable
to the previous bounds on the other side of the variable. Since bounds on
opposite sides are the only opportunities for contradictions, the requirement
preserves the strengthened substance property from the induction hypothesis.

\TODO{note how substance allows positive packing to place generated constraints into qualified types without violating soundness }

\begin{theorem} 
  \label{thm:proof_subtyping_soundness}
  \emph{proof subtyping soundness} 
  \small
  \nopad
  \begin{mathpar}
    \inferrule {
      \Theta, \Delta \entails \tau_l \subtypes \tau_r \given \Theta', \Delta'
    } {
      \exists \delta' \qua 
      \textbf{dom}(\delta') \subseteq \Theta' \backslash \Theta
      \up
      (
      \forall \delta \qua 
      \delta \oplus \delta' \satisfies \Delta' \implies
      \delta \oplus \delta' \satisfies \tau_l \subtypes \tau_r
      )
    }
  \end{mathpar}
\end{theorem}

\noindent
Proof subtyping soundness states that if proof subtyping generates subtyping constraints \ms{\Delta'} 
for a subtyping goal \ms{\tau_l \subtypes \tau_r},
then any solution for the generated subtyping constraints
is a solution for the subtyping goal, with the caveat that
it only needs to hold for some interpretation of the generated skolems \ms{\Theta'\backslash\Theta}. 
The proof is by induction, leveraging numerous lemmas to derive the semantics
defined by model typing in each case. 
Completeness of restricted forms is necessary whenever the corresponding model semantics 
is used in a negative position, such as in negation or in the antecedent of an implication.     
Additionally, when rewriting an object, it is necessary to characterize how the model
semantics of its input form relates to the model semantics of its output form.
Lastly, when pruning subtyping constraints generated by the solver, it is necessary
that the remaining subtyping are as strong or stronger that the removed subtyping constraints
with respect to the variables used in the goal.
The property of proof subtyping soundness is a bit weak on its own. 
It does not guarantee that there is a solution for the subtyping goal,
it could be met vacuously by having proof subtyping 
generate an unsatisfiable subtyping constraints.
To make the stronger guarantee, we can combine it with another theorem.

\begin{theorem}
  \label{thm:proof_typing_substance}
  \emph{proof typing substance} 
  \small
  \nopad
  \begin{mathpar}
    \inferrule {
      \Theta, \Delta, \Gamma \entails e \hastype \tau \given \Theta', \Delta'
    } {
      \forall \delta \qua \textbf{dom}(\delta) \subseteq \Theta \implies
      \exists \delta' \qua \delta \oplus \delta' \satisfies \Delta 
      \implies
      \exists \delta' \qua \delta \oplus \delta' \satisfies \Delta'
    }
  \end{mathpar}
\end{theorem}

\noindent
Proof typing substance states that the generated subtyping constraints are satisfiable
assuming the contextual constraints are satisfiable, with respect
to all interpretations of the contextual skolems.   
The proof is by induction, and it follows easily from the induction hypothesis
and proof subtyping soundness. 

\begin{theorem}
  \label{thm:proof_typing_soundness}
  \emph{proof typing soundness} 
  \small
  \nopad
  \begin{mathpar}
    \inferrule {
      \Theta, \Delta, \Gamma \entails e \hastype \tau \given \Theta', \Delta'
    } {
      \exists \delta' \qua \textbf{dom}(\delta') \subseteq \Theta' \backslash \Theta \up 
      (
      \forall \delta \qua
      \delta \oplus \delta' \satisfies \Delta' \implies
      \delta, \sigma \satisfies \Gamma \implies
      \delta \oplus \delta' \satisfies e[\sigma] \hastype \tau
      )
    }
  \end{mathpar}
\end{theorem}

\noindent
Proof typing soundness states that if proof typing generates subtyping constraints \ms{\Delta'} 
for a typing goal \ms{e \subtypes \tau},
then any solution for the contextual typing constraints 
and the generated subtyping constraints
is a solution for the typing goal, with the caveat that
it only needs to hold for some interpretation of the generated skolems \ms{\Theta'\backslash\Theta}. 
We prove soundness by induction, leveraging subtyping soundness,
and additional lemmas the characterize the relationships between rewritten forms of objects. 
Proof typing soundness as the same limitations as proof subtyping soundness. 
It's not strong enough to guarantee our goal of type inference,
but we can easily compose it with another property to do so.

% \begin{theorem}
%   \label{prop:proof_typing_annotation_soundness}
%   \emph{proof typing annotation soudness}
%   \\
%   \begin{mathpar}
%     \inferrule {
%       \Theta, \Delta, \Gamma \entails \J{def } x \J{:} \tau_a \J{ = } e \J{ in } e' \hastype \tau' \given \Theta', \Delta'
%     } {
%       \forall \delta, \sigma \qua 
%       \delta, \sigma \satisfies \Gamma \implies
%       \delta \satisfies e \hastype \tau_a
%     }
%   \end{mathpar}
% \end{theorem}

% \noindent
% In addition to inferring type according to the control flow of a program,
% we are also interested in checking that programs meet specifications 
% annotated by the programmer. Proof typing soundness 
% does not directly reference these annotations, 
% so we need an additional theorem to ensure
% the annotations are respected. 


\section{Experiments}
\label{sec:experiments}

\noindent
We performed experiments to see how type inference performs on both functions and constructors. 
These two categories have historically been tackled independently of each other, where parametric types
tend to rely on notions of instantiation, and structural types rely on notions of equality or subtyping.
Our system unifies these historically distinct problems by describing both kinds of types in terms of subtyping.
However, the distinction is useful for comparing our design to previous work. 

We implemented an extrinsic type inference system that corresponds to the predicate definitions
presented thus far. 
To observe the expressiveness of our system on functions, 
we evaluated our implementation on the same suite of programs 
that SuperF was evaluated on, as presented in the paper on SuperF \cite{}.
Based on these experiments, our system all programs admitted by SuperF, as well as some programs 
that SuperF is unable to admit.
Thus, our system appears to be more expressive on functions than the previous most expressive
type inference system that we know of.
To observe the expressiveness of our system on constructors, 
we constructed simple programs that represent the kind of flexible and reusable code
often found programs written in untyped scripting languages. 
We ran these examples in our implementation, and created analogous version to run
in the online implementation of MLScript \cite{},
which represents the theoretical system MLStruct \cite{}. 
All of these examples are admitted by our system, while MLStruct
can either requires intrinsic specifications to admit the program, 
or simply can't admit the program.



\begin{definition} 
  \label{def:experimental_expressions_and_types}
  \emph{experimental expressions and types} 
  \scriptsize
  \nopad
  \[
  \begin{array}[t]{ll}
    \begin{array}[t]{r@{}c@{}l}
      \textbf{Bool} 
      &\ \ =\ \ & 
      \J{(<true>@ | <false>@)} 
      \\
      \textbf{Nat} 
      &\ \ =\ \ & 
      \J{(LFP[R] <zero>@ | <succ>R)} 
      \\
      \textbf{Even} 
      &\ \ =\ \ & 
      \J{(LFP[R] <zero>@ | <succ><succ>R)} 
      \\
      \textbf{List}(\alpha) 
      &\ \ =\ \ & 
      \J{(LFP[R] <nil>@ | <cons>((}\alpha\J{)*R))} 
      \\
      \textbf{ChurchNum} 
      &\ \ =\ \ & 
      \J{(ALL[T] (T -> T) -> T -> T)}
    \end{array}
    &
    \begin{array}[t]{r@{}c@{}l}
      \textbf{ST}(\alpha^\dagger, \alpha) 
      &\ \ =\ \ & 
      \J{(<tag>(}\alpha^\dagger \J{) <result>(} \alpha \J{))}
      \\
      \textbf{z} 
      &\ \ =\ \ & 
      \J{[f => [x => x]]} 
      \\
      \textbf{s} 
      &\ \ =\ \ & 
      \J{[n => [f => [x => f(n(f(x)))]]]} 
      \\
      \textbf{n3} 
      &\ \ =\ \ & 
      \textbf{s}\J{(}\textbf{s}\J{(}\textbf{s}\J{(}\textbf{z}\J{)}\J{)}\J{)} 
    \end{array}
  \end{array}
  \]
\end{definition}

\noindent
For clarity, we define a few metalinguistic terms to stand in for 
types and expressions used as inputs to our system. 


\begin{definition} 
  \label{def:experimental_typing_context}
  \emph{experimental typing context}
  \tiny
  \nopad
  \[
  \begin{array}[t]{ll}
    \begin{array}[t]{r@{}c@{}l}
      \textbf{ctx} & = &  \hfill
      \\
      \J{head} &\J{ : }& \J{ALL[T] }\textbf{List}(\J{T})\J{ -> T}
      \\
      \J{tail} &\J{ : }& \J{ALL[T] }\textbf{List}(\J{T})\J{ -> }\textbf{List}(\J{T})
      \\
      \J{nil} &\J{ : }& \J{ALL[T] }\textbf{List}(\J{T})
      \\
      \J{single} &\J{ : }& \J{ALL[T] T -> }\textbf{List}(\J{T})
      \\
      \J{append} &\J{ : }& \J{ALL[T] }\textbf{List}(\J{T}) \J{ -> } \textbf{List}(\J{T}) \J{ -> } \textbf{List}(\J{T})
      \\
      \J{length} &\J{ : }& \J{ALL[T] }\textbf{List}(\J{T}) \J{ -> } \textbf{Nat}
      \\
      \J{const} &\J{ : }& \J{ALL[T U] T -> U -> T}
      \\
      \J{id} &\J{ : }& \J{ALL[T] T -> T}
      \\
      \J{ids} &\J{ : }& \textbf{List}(\J{ALL[T] T -> T}\J{)}
      \\
      \J{inc} &\J{ : }& \textbf{Nat}\J{ -> }\textbf{Nat}
      \\
      \J{choose} &\J{ : }& \J{ALL[T] T -> T -> T}
      \\
      \J{poly} &\J{ : }& \J{(ALL[T] T -> T) -> } \textbf{Nat}\J{*}\textbf{Bool}
      \\
      \J{fst} &\J{ : }& \J{ALL[T U] T*U -> T}
      \\
      \J{sort} &\J{ : }& \J{ALL[T] (T * T -> } \textbf{Bool} \J{) -> }\textbf{List}(\J{T})\J{ -> }\textbf{List}(\J{T})
      \\
      \J{scalarCmp} &\J{ : }& \textbf{Nat} \J{\*} \textbf{Nat} \J{ -> } \textbf{Bool} 
      \\
      \J{lexicoCmp} &\J{ : }& \textbf{List}(\textbf{Nat}) \J{\*} \textbf{List}(\textbf{Nat}) \J{ -> } \textbf{Bool} 
    \end{array}
    &
    \begin{array}[t]{r@{}c@{}l}
      \\
      \J{auto} &\J{ : }& \J{(ALL[T] T -> T) -> (ALL[T] T -> T)}
      \\
      \J{auto'} &\J{ : }& \J{ALL[U] (ALL[T] T -> T) -> U -> U}
      \\
      \J{map} &\J{ : }& \J{ALL[T U] (T -> U) -> } \textbf{List}(\J{T}) \J{ -> } \textbf{List}(\J{U})
      \\
      \J{app} &\J{ : }& \J{ALL[T U] (T -> U) -> T -> U}
      \\
      \J{revapp} &\J{ : }& \J{ALL[T U] T -> (T -> U) -> U}
      \\
      \J{runST} &\J{ : }& \J{ALL[T] (ALL[U] } \textbf{ST}(\J{U}, \J{T})\J{) -> T}
      \\
      \J{argST} &\J{ : }& \J{ALL[U] } \textbf{ST}(\J{U}, \textbf{Nat})
      \\
      \J{zero} &\J{ : }& \textbf{ChurchNum}
      \\
      \J{succ} &\J{ : }& \textbf{ChurchNum} \J{ -> } \textbf{ChurchNum}
      \\
      \J{foo} &\J{ : }& \J{ALL[T] (T -> T) -> } \textbf{List}(\J{T}) \J{ -> T}
      \\
      \J{g} &\J{ : }& \J{ALL[T] }\textbf{List}(\J{T}) \J{ -> } \textbf{List}(\J{T}) \J{ -> T}
      \\
      \J{k} &\J{ : }& \J{ALL[T] T -> } \textbf{List}(\J{T}) \J{ -> T}
      \\
      \J{h} &\J{ : }& \textbf{Nat} \J{-> ALL[T] T -> T}
      \\
      \J{l} &\J{ : }& \textbf{List}(\J{ALL[T] } \textbf{Nat} \J{ -> T -> T}\J{)}
      \\
      \J{r} &\J{ : }& \J{(ALL[T] T -> ALL[U] B -> B) -> }\textbf{Nat}
    \end{array}
  \end{array}
  \]
\end{definition}

\noindent
Additionally, for the purpose of experiment, it is useful to assume that   
the typing of some identifiers have already been learned or verified.

\begin{experiment}
  \label{exp:part_1}
  Something
  \hfill
  \scriptsize
  \nopad
  \begin{center}
  \begin{tabular}{l m{30em} >{\centering}m{5em} >{\centering}m{5em} >{\centering\arraybackslash}m{5em} } 
    \multicolumn{5}{l}{
      \small
      Does the implementation admit an expression under typing constraints \textbf{ctx}?
    } \\

    \multicolumn{4}{l}{
      \small
      \pass\ yes \ \ $\cdot$\ \ \fail\ no \ \ $\cdot$\ \ \assisted\ conditionally w/ annotations 
    } \\

    \stoprule

    \multicolumn{2}{c}{} & \multicolumn{3}{c}{\small implementation} \\ 

    \scmidrule{3-5}

    \multicolumn{2}{c}{\small expression} & \textbf{Extrinsic} & \textbf{SuperF} & \textbf{SF no D} \\ 

    \smidrule
    \pdc. &
    \J{[x => [y => y]]} 
    & \pass & \pass & \pass \\

    \sline

    \pdc. &
    \J{choose(id)} 
    & \pass & \pass & \pass \\

    \sline

    \pdc. &
    \J{choose(nil)(id)} 
    & \pass & \pass & \pass \\

    \sline

    \pdc. &
    \J{[x => x(x)]} 
    & \pass & \pass & \pass \\

    \sline

    \pdc. &
    \J{id(auto)} 
    & \pass & \pass & \pass \\

    \sline

    \pdc. &
    \J{id(auto')} 
    & \pass & \pass & \pass \\

    \sline

    \pdc. &
    \J{choose(id)(auto)} 
    & \pass & \pass & \pass \\

    \sline

    \pdc. &
    \J{choose(id)(auto')} 
    & \pass & \pass & \pass \\

    \sline

    \pdc. &
    \J{foo(choose(ids))(ids)} 
    & \pass & \pass & \pass \\

    \sline

    \pdc. &
    \J{poly(id)} 
    & \pass & \pass & \pass \\

    \sline

    \pdc. &
    \J{poly([x => x])} 
    & \pass & \pass & \pass \\

    \sline

    \pdc. &
    \J{id(poly)([x => x])} 
    & \pass & \pass & \pass \\

    \sline

    \pdc. &
    \J{[f => f(succ;zero;@), f(true;@)]}
    & \pass & \pass & \pass \\

    \sline

    \pdc. &
    \J{[xs => poly(head(xs))]}
    & \pass & \pass & \pass \\

    \sbottomrule
  \end{tabular}
  \end{center}
\end{experiment}

\noindent
In the experiment on functions, we 
are interested to observe if our implementation (\textbf{Extrinsic})
can admit various programs on functions. 
For our implementation, we reject a program if the implementation
finds a contradiction, diverges, or infers the top type. 
In our system, we infer a top type when the top level 
expression represents a function and all of its branches 
have errors. Since the top level function is not applied,
the program technically cannot produce an error, but it's 
perfectly useless and would produce an error if applied to any value. 
We compare our system to SuperF and Super without Distributivity (\textbf{SF no D}). 

% $\approx \exists \tau, \Delta, \Theta \qua \epsilon, \epsilon, \textbf{ctx} \entails e \hastype \tau \given \Theta, \Delta$ 

\begin{experiment}
  \label{experi:part_2}
  \hfill
  \scriptsize
  \nopad
  \begin{center}
  \begin{tabular}{l m{30em} >{\centering}m{5em} >{\centering}m{5em} >{\centering\arraybackslash}m{5em} } 
    \multicolumn{5}{l}{
      \small
      Does the implementation admit an expression under typing constraints \textbf{ctx}?
    } \\

    \multicolumn{4}{l}{
      \small
      \pass\ yes \ \ $\cdot$\ \ \fail\ no \ \ $\cdot$\ \ \assisted\ conditionally w/ annotations 
    } \\

    \stoprule

    \multicolumn{2}{c}{} & \multicolumn{3}{c}{\small implementation} \\ 

    \scmidrule{3-5}

    \multicolumn{2}{c}{\small expression} & \textbf{Extrinsic} & \textbf{SuperF} & \textbf{SF no D} \\ 

    \smidrule

    \pdc. &
    \J{length(ids)}
    & \pass & \pass & \pass \\

    \sline
    
    \pdc. &
    \J{tail(ids)}
    & \pass & \pass & \pass \\

    \sline
    
    \pdc. &
    \J{head(ids)}
    & \pass & \pass & \pass \\

    \sline
    
    \pdc. &
    \J{single(id)}
    & \pass & \pass & \pass \\

    \sline
    
    \pdc. &
    \J{cons(id)(ids)}
    & \pass & \pass & \pass \\

    \sline
    
    \pdc. &
    \J{cons([x => x])(ids)}
    & \pass & \pass & \pass \\

    \sline
    
    \pdc. &
    \J{append(single(inc))(single(id))}
    & \pass & \pass & \pass \\

    \sline
    
    \pdc. &
    \J{g(single(id))(ids)}
    & \pass & \pass & \pass \\

    \sline
    
    \pdc. &
    \J{map(poly)(single(id))}
    & \pass & \pass & \pass \\

    \sline
    
    \pdc. &
    \J{map(head)(single(ids))}
    & \pass & \pass & \pass \\

    \sline
    
    \pdc. &
    \J{app(poly)(id)}
    & \pass & \pass & \pass \\

    \sline
    
    \pdc. &
    \J{revapp(id)(poly)}
    & \pass & \pass & \pass \\

    \sline
    
    \pdc. &
    \J{runST(argST)}
    & \pass & \pass & \pass \\

    \sline
    
    \pdc. &
    \J{app(runST)(argST)}
    & \pass & \pass & \pass \\

    \sline
    
    \pdc. &
    \J{revapp(argST)(runST)}
    & \pass & \pass & \pass \\

    \sbottomrule
  \end{tabular}
  \end{center}
\end{experiment}

\begin{experiment}
  \label{experi:part_3}
  \hfill
  \scriptsize
  \nopad
  \begin{center}
  \begin{tabular}{l m{30em} >{\centering}m{5em} >{\centering}m{5em} >{\centering\arraybackslash}m{5em} } 
    \multicolumn{5}{l}{
      \small
      Does the implementation admit an expression under typing constraints \textbf{ctx}?
    } \\

    \multicolumn{4}{l}{
      \small
      \pass\ yes \ \ $\cdot$\ \ \fail\ no \ \ $\cdot$\ \ \assisted\ conditionally w/ annotations 
    } \\

    \stoprule

    \multicolumn{2}{c}{} & \multicolumn{3}{c}{\small implementation} \\ 

    \scmidrule{3-5}

    \multicolumn{2}{c}{\small expression} & \textbf{Extrinsic} & \textbf{SuperF} & \textbf{SF no D} \\ 

    \smidrule
    
    \pdc. &
    \J{k(h)(l)}
    & \pass & \pass & \pass \\

    \sline
    
    \pdc. &
    \J{k([x => h(x)])(l)}
    & \pass & \pass & \pass \\

    \sline
    
    \pdc. &
    \J{r([x => [y => y]])}
    & \pass & \pass & \pass \\

    \sline
    
    \pdc. &
    \J{auto(id)}
    & \pass & \pass & \pass \\

    \sline
    
    \pdc. &
    \J{cons(head(ids))(ids)}
    & \pass & \pass & \pass \\

    \sline
    
    \pdc. &
    \J{head(ids)(succ;succ;succ;zero;@)}
    & \pass & \pass & \pass \\

    \sline
    
    \pdc. &
    \J{choose(head(ids))}
    & \pass & \pass & \pass \\

    \sline
    
    \pdc. &
    \J{def f = revapp(id) in f(poly)}
    & \pass & \pass & \pass \\

    \sline
    
    \pdc. &
    \J{choose(id)([x => auto'(x)])}
    & \pass & \pass & \pass \\

    \sbottomrule
  \end{tabular}
  \end{center}
\end{experiment}

\begin{experiment}
  \label{experi:part_4}
  \hfill
  \nopad
  \scriptsize
  \begin{center}
  \begin{tabular}{l m{30em} >{\centering}m{5em} >{\centering}m{5em} >{\centering\arraybackslash}m{5em} } 
    \multicolumn{5}{l}{
      \small
      Does the implementation admit an expression under typing constraints \textbf{ctx}?
    } \\

    \multicolumn{4}{l}{
      \small
      \pass\ yes \ \ $\cdot$\ \ \fail\ no \ \ $\cdot$\ \ \assisted\ conditionally w/ annotations 
    } \\

    \stoprule

    \multicolumn{2}{c}{} & \multicolumn{3}{c}{\small implementation} \\ 

    \scmidrule{3-5}

    \multicolumn{2}{c}{\small expression} & \textbf{Extrinsic} & \textbf{SuperF} & \textbf{SF no D} \\ 

    \smidrule
    
    \pdc. &
    \J{def ignore : } \textbf{ChurchNum} \J{ = z in @}
    & \pass & \pass & \pass \\

    \sline
    
    \pdc. &
    \J{s}
    & \pass & \pass & \pass \\

    \sline
    
    \pdc. &
    \J{def ignore : } \textbf{ChurchNum} \J{ -> } \textbf{ChurchNum} \J{ = s in @}
    & \pass & \pass & \fail \\

    \sline
    
    \pdc. &
    \J{def ignore : } \textbf{ChurchNum} \J{ = n3 in @}
    & \pass & \pass & \pass \\

    \sline
    
    \pdc. &
    \J{def ignore : } \J{@ -> }\textbf{ChurchNum} \J{ = [@ => n3(n3)] in @}
    & \pass & \pass & \fail \\

    \sline
    
    \pdc. &
    \J{fst(fst(fst(n3([x => x,(zero;@)])(succ;zero;@))))}
    & \pass & \pass & \pass \\

    \sline
    
    \pdc. &
    \J{(s(s(z)))(s(s(z)))}
    & \pass & \pass & \pass \\

    \sline
    
    \pdc. &
    \cb{
      \J{def to\_church = loop([self => } 
      \\
      \I \J{[zero;@ => z]}
      \\
      \I \J{[succ;n => s(self(n))]}
      \\
      \J{]) in ...}
    }
    & \pass & \pass & \pass \\

    \sline
    
    \pdc. &
    \J{def ignore : } \textbf{Nat} \J{ -> } \textbf{ChurchInt} \J{ = to\_church in @}
    & \pass & \pass & \fail \\

    \sline
    
    \pdc. &
    \J{loop([self => [x => if true;@ then x else self(self)(x)]]} 
    & \fail & \fail & \fail \\

    \sline
    
    \pdc. &
    \J{[x => x(x)]([x => x(x)])}
    & \pass & \fail & \fail \\

    \sline
    
    \pdc. &
    \J{auto(auto'(id))}
    & \pass & \pass & \fail \\

    \sline
    
    \pdc. &
    \cb{
      \J{[y => } 
      \\
      \I \J{def tmp = y in}
      \\
      \I \J{y(const)}
      \\
      \J{][x => x(x)]}
    }
    & \fail & \pass & \pass \\

    \sline
    
    \pdc. &
    \cb{
      \J{[k => (k([x => x])),(k([x => single(x)])) ](}
      \\
      \I \J{[f => (f(succ;zero;@)),(f(true;@))]}
      \\
      \J{)}
    }
    & \pass & \pass & \pass \\

    \sline
    
    \pdc. &
    \cb{
      \J{[f => } 
      \\
      \I \J{def a = [@ => f(id)] in }
      \\
      \I \J{def y : } \textbf{Nat} \J{ -> (ALL[T] T -> T) = a(@) in }
      \\
      \I \J{y}
      \\
      \J{](const(const(id)))}
    }
    & \fail & \fail & \pass \\

    \sbottomrule
    
  \end{tabular}
  \end{center}
\end{experiment}


\begin{experiment}
  \label{experi:part_5}
  \hfill
  \nopad
  \scriptsize
  \begin{center}
  \begin{tabular}{p{1em} l >{\centering}p{5em} >{\centering\arraybackslash}p{5em}} 
    \multicolumn{4}{l}{
      \small
      Does the implementation admit an expression under typing constraints \textbf{ctx}?
    } \\

    \multicolumn{4}{l}{
      \small
      \pass\ yes \ \ $\cdot$\ \ \fail\ no \ \ $\cdot$\ \ \assisted\ conditionally w/ annotations 
    } \\

    \stoprule

    \multicolumn{2}{c}{} & \multicolumn{2}{c}{\small implementation} \\ 

    \scmidrule{3-4}

    \multicolumn{2}{c}{\small expression} & \textbf{Extrinsic} & \textbf{MLStruct} \\ 

    \smidrule

    \sdc. &
    \cb{
      \J{def stdCmp = [ a,b => } 
      \\
      \I \J{a |> (}
      \\
      \I\I \J{[zero;@ => scalarCmp(a,b)]}
      \\
      \I\I \J{[succ;n => scalarCmp(a,b)]}
      \\
      \I\I \J{[nil;@ => lexicoCmp(a,b)]}
      \\
      \I\I \J{[cons;(x,xs) => lexicoCmp(a,b)]}
      \\
      \I \J{)}
      \\
      \J{] in ...}
    }
    & \pass & \assisted \\

    \sline

    \sdc. &
    \cb{
      \J{def stdSort : (TOP } 
      \\
      \I \J{ \& (} \textbf{List}(\textbf{Nat}) \J{ -> } \textbf{List}(\textbf{Nat}) \J{)}
      \\
      \I \J{ \& (} \textbf{List}(\textbf{List}(\textbf{Nat})) \J{ -> } \textbf{List}(\textbf{List}(\textbf{Nat})) \J{)}
      \\
      \J{) = sort(stdCmp) in ...} 
    }
    & \pass & \assisted \\

    \sline

    \sdc. &
    \cb{
      \J{def double : } \textbf{Nat} \J{ -> } \textbf{Even} \J{ = loop([self => } 
      \\
      \I \J{[zero;@ => zero;@]}
      \\
      \I \J{[succ;n => succ;succ;(self(n))]}
      \\
      \J{]) in ...}
    }
    & \pass & \fail \\
    
    \sline

    \sdc. &
    \cb{
      \J{def halve : } \textbf{Even} \J{ -> } \textbf{Nat} \J{ = loop([self => } 
      \\
      \I \J{[zero;@ => zero;@]}
      \\
      \I \J{[succ;succ;n => succ;(self(n))]}
      \\
      \J{]) in ...}
    }
    & \pass & \fail \\
    
    \sline

    \sdc. &
    \cb{
      \J{[x => halve(double(x))]}
    }
    & \pass & \fail \\

    \sline

    \sdc. &
    \cb{
      \J{[zero;@ => [zero;@ => @](@)]}
      \\
      \J{[nil;@ => @]}
    }
    & \pass & \fail \\

    \sbottomrule
    
  \end{tabular}
  \end{center}
\end{experiment}


\TODO{add more interesting examples and explain why they're interesting}

\TODO{add example of of inductive reasoning to prove that an even number is not three}

\TODO{add example of path sensitivity for if then else conditional}

\TODO{add example of a counter stream}

\noindent
The experiment on programs with constructors demonstrate
that our system is roughly as expressive as untyped languages
without sacrificing safety guarantees.
Our system can easily mix structures together, reducing
the need for programmers to call different interfaces for different data.  
If there is an error in one branch of a function, the
system can still admit the program by ensuring 
that the erroneous branch is never used, and there
is at least one branch that could be used.


\section{Related Work}
\label{sec:related_work}

Our system approximates values with many forms of types
and reasons over their control with with subtyping. Our work
builds on the techniques from MLStruct \cite{} and SuperF \cite{}, which
like our system must reason over the combination of numerous forms of types.
Leading up to these systems, there have been numerous techniques developed
to reason over subsets of these type forms, with varying degrees 
of intrinsic typing. 

% Over the past decades many type inference have been designed and analyzed 
% with varying degrees of expressivity and varying degrees of intrinsic requirements
% or extrinsic relaxations, with respect to both functions (e.g. functions or paths dependent on variables)
% and constructors (e.g. data constructors or records).  

The first technique to infer parametric types without any assistance 
from annotations was Hindley-Milner type inference (HM)\cite{}. 
Despite its age, it remains one of the foundational 
theoretical underpinnings of practical programming languages used today, such as Standard ML \cite{}. 
For instance, in Standard ML, the system infers the identifier \code{foo} in \code{fun foo f = (f 123)}
to have type \code{forall 'a . (int -> 'a) -> 'a}
Despite its practical utility, a significant drawback of HM is that
certain kinds of general-purpose expressions are simply not allowed.
For instance, in Standard ML, the system rejects the program \code{fun fooPair f = (f 123, f true)}.

In order to type the program \code{fooPair}, a type system would either need to determine that
\code{f} is parametric, or that the types of \code{123} and \code{true} are compatible.
Treating the parameter \code{f} is allowed in systems that have impredicative parametric polymorphism \cite{}, 
which was introduced by System F \cite{}.


To reduce the annotation burden and increase expressivity, researchers
developed ways to infer type instantiations and compare types with varying
multiple possible instantiations, by combining impredicative types
with type inference techniques and ways to compare approximations of varying precision . 

One of possibility is bidirectional type inference \cite{}  
is one such technique where bounds on type variables can be inferred in 
both directions of the control flow.
It propagates inferred and annotated knowledge directly from
the structures of the program and relies on subtyping merely to check compatibility between
types. Additionally, whether a non-parametric type is an instantiation
of a parametric type is treated as distinct from subtyping. 

In order to maintain HM's implicit first-order types in the context
of higher-order types, other techniques were explored.  
These techniques include boxed polymorphism \cite{}, ML$^F$ \cite{}, and various other systems \cite{}. 
Unfortunately, it's not clear what parametric type should be inferred for \code{f}.
There are an infinite number of
parametric types that could be used, but none is weaker than all the rest.

\examp{
  \J{forall 'a . 'a -> 'a}
  \\
  \J{forall 'a . 'a -> 'a list}
  \\
  \J{forall 'a . 'a -> 'a option}
}

Thus, these systems with impredicative parametric polymorphism rely on annotations of parameters to
determine second order parametric types, which constitutes intrinsic typing of
of functions. 
Since there is no parameter type that is weaker than all the rest,
merely relying on intrinsic second order parametric types 
significantly restricts the reusability of expressions.
Additionally, many of these techniques require explicit instantiation to use parametrically typed
expressions, which can be seen as an intrinsic restriction that 
delays pruning the control-flow until a particular usage. 





All the type inference techniques depend reasoning over the observed control flow  
of programs, but the reasoning to particular situations and objects and use different
relations for solving vs checking.  The subtyping relation was generalized in the theory of implicit coercions (Cretin) \cite{},
such that even instantiation is treated as subtyping (or implicit coercion).  
Based on these ideas, SuperF \cite{} introduced type inference techniques for generating actually solving
these generalized subtyping constraints

SuperF \cite{} introduced 
a technique that relies on extrinsic subtyping polymorphism, 
which bypasses the need for intrinsic parametric polymorphism.
In the previous example, SuperF can infer the principal type for \code{fooPair} by inferring the 
weakest type of its parameter \code{f}, which it achieved by leveraging subtyping to find 
a type that is compatible with both \code{123} and \code{true}.
The SuperF type system infers subtyping constraints that represent the data flow represented by
function application and pattern matching. It relies 
intersection and union type combinators to represent the strongest or weakest interpretations
of type variables that meet certain subtyping constraints.   
With SuperF's type inference mechanism, it is possible to infer the principal type for the 
previous problematic example \code{fooPair: (forall 'a 'b . (int -> 'a) /\textbackslash\ (bool -> 'b) -> 'a * 'b)}.

SuperF has a restricted language of types. It is concerned with the type inference
of functions, and it is agnostic towards type inference of constructors. 
It restricts the type and subtyping syntax according to polarity; that is, a
types position within an implication type or subtyping relation. Depending on polar position,
universal types (i.e. universals) may or may not be qualified by subtyping constraints.
The authors of SuperF highlight some technical features, including extrusion and distributivity.
Their extrusion technique ensures that types are not overly specialized
when inferring types of nested functions. Their distributivity technique allows
rewriting universal quantifiers to increase the chance that skolemization occurs before instantiation. 
They point out that the distributivity rule is an improvement for most cases, but there are certain
problems for which it more actually more restrictive.


The high level ideas in SuperF of relying on subtyping constraints were very influential in our work.
Like SuperF, our system relies on syntax-directed subtyping rules to check constraints and to learn 
new subtyping constraints by traversing previously learned constraints;
Additionally, our system also relies on instantiation and skolemization in its subtyping rules. 
Type polarity is also important in our system.
However, our system does not restrict the syntax of types based on polarity. Instead, we rely on
polarity to constrain proof typing and proof subtyping when interpreting or checking interpretations
of type variables.    
Our system does not need extrusion because Packing \ms{\textbf{pack(...)}} preserves the nested structure of 
functions at the type level, such that the constraint between inner and outer bound type variables
qualifies the inner parametric type, which ensures that instantiation of the inner parametric type 
continues to constrain the outer bound type variables.  
Our system does not need distributivity of type variable bindings, because it requires
that skolemization occurs before learning constraints on type variables. Therefore, it is fine to allow
instantiation to occur before skolemization in our system, 
since it does not cause learning constraints before skolemization.  

\TODO{look into what SuperF folks mean by no backtracking in light of rhs union}

\TODO{backtracking is necesary to infer strongest possible functions (all the cases)}

\TODO{therefore, they must mean they avoid redundant backtracking, that mearely learns weaker constraints}


\TODO{recursive types only matter in light of structures}



In addition to functions, practical programming languages need to have some form of constructors.
Although Hindley-Milner type inference offered an extrinsic type inference solution for predicative functions
very early on, type inference has always relied on some degree of intrinsic types for constructors.
intrinsically typed constructors is exemplified by the datatype mechanism of Standard ML.
For instance, the list datatype \code{'a list = nil | cons 'a * 'a list} specifies that 
data constructed with \code{nil} or \code{cons} have the type \code{'a list}. Additionally,
it requires that the input to the constructor \code{cons}  has the types \code{'a * 'a list}
Although the datatype mechanism has been used with great success across many languages (including Lean, Haskell, Ocaml, and Isabelle),
it enforces a very rigid form of programming. For instance, there is no way to guarantee
that the output of a function is a nonempty list and then use that output in a function that expects an optionally empty list.

To get around this limitation, refinement types \cite{} relaxed the meaning of the datatype mechanism, to 
allow constructions to represent types that are stronger than their intrinsic type.
For instance, \code{nil} could have a type as strong as the single type \code{?nil} or as weak as \code{'a list}.
Likewise, a list of two elements \code{cons(x, cons(y, nil))} could have a type as strong 
\code{?cons 'a * (?cons 'a * nil)}
or as weak as \code{'a list}. 
A refinement type system can choose how strong to make the type based on how the data is used, 
which depends on how data flows, for which subtyping provides a natural formulation. 
It can find these stronger types by starting with intrinsic weakest type and using intersection with 
all the ways it sees the data being used.
However, the refinement relaxation is not enough to allow common flexible idioms used in untyped languages. 
For instance, it would allow pattern matching that chooses between a list and a number.


To allow arbitrary mixing of constructions in pattern matching, the extrinsic type of constructions
must be abolished so that there is no intrinsic weakest type other than a top type. 
In MLSub and algebraic subtyping \cite{}, forgoes the datatype mechanism.
Instead, it uses type inference to generates subtyping constraints and solve them, akin to unification in HM.

Structures, when combined with loops, enable the construction of nested data (i.e. trees) 
of arbitrary size. To ensure recursive functions on trees are error-free,
systems have used regular tree grammars and recursive types to characterize spaces of trees.
There has been plenty of research on reasoning over regular tree grammars/automata and recursive types \cite{}.
Most relevant to our system is previous work on subtyping of recursive types \cite{}.  

\TODO{compare inductive subtyping to tree interpolation}

\TODO{is this something we should/can prove?}

\TODO{note how CHC uses predefined constraints, while this system learns constraints from applications}

\TODO{other systems (using CHC) can construct recursive horn clauses from programs, but not local constraints, right?}

\TODO{that is the language of recursive types is used as both the constraint language and the target language}

\TODO{understand the capabilities and limitations of MLSub better}

Popular scripting languages have also been augmented with some type inference capabilities.
Typescript \cite{} does not have an intrinsic weakest type for constructions,
and it can infer output types of functions that return different structural forms in different branches.
However, it is unable to infer a weak type for the parameters, and thus requires annotation of parameters.  
Typescript and other systems for scripting languages, rely on flow-sensitive, or occurrence typing \cite{},
which can refine the types of variables in branches based on the condition guarding each branch.
For Typescript, this flow information is only used in one direction, and it is not leveraged to infer
the weakest type of parameters.

To extend type inference of scripting languages further, MLStruct \cite{} and others like MLSub \cite{}, 
leveraged the flow information to infer both input types and output types of functions.
MLStruct, along with SuperF, represents some of the foundational ideas of the MLScript language.
Like SuperF, it learns the data flow of programs in terms of subtyping constraints.
Other than our system, it appears that MLStruct has the best type inference of constructors
with the least amount of intrinsic type requirements. 
However, MLStruct still has a significant intrinsic type requirement, which reduces flexibility compared
to untyped languages. It requires that the inputs to data constructors have an intrinsic type.  
In essence, this requirement prescribes a lower bound on data constructions,
which reduces the flexibility of programs compared to untyped languages. 
As our tests (Experiment \ref{experi:typing_structures})
illustrate, it is not possible to type a function that expects only even numbers that is interoperable with natural numbers.
Moreover, the additional annotation burden alone would prevent the analysis of untyped code without
significant effort in augmenting the code with annotations.
In contrast to MLStruct, our system relies completely on learning data flow in the form of subtyping constraints
without any annotation requirements on constructors. 

\TODO{note work on recursive types and regular trees}

The common ingredient in all of these systems that perform some amount of extrinsic type inference is
learning data flow. Data flow analysis comes in many forms. As the goal for these systems is verification,
subtyping is a natural choice. There has been some work on extracting rules for checking subtyping relations
from the semantic interpretation of types, known as semantic subtyping \cite{}, or set-theoretic types \cite{}.
Other approaches, like algebraic subtyping \cite{}, and Cretin \cite{}, and superF, 
rely on designing syntax-directed subtyping rules and then proving their soundness with respect to some semantic interpretation. 
Additionally, there are similarities between type inference with subtyping and control-flow analysis \cite{}, or polyvariant flow analysis \cite{}.
In control-flow analysis, expressions can be associated with a label, and each label can map to a set of abstract values,
determined by learning subset constraints from the control-flow of the program. 
Instead of labels, type inference with subtyping associate types  
with expressions and learns subtyping constraints.

\section{Conclusion}
\label{sec:conclusion}
We have presented a novel extrinsic type inference system
that can guarantee programs are bug-free while maintaining
code-reusability nearly on par with untyped scripting languages,
and exceeding that of an previous type inference system we know of.  
We have incorporated previous studied ideas of subtyping constraints,
intersection types, union types, parametric types, and recursive types into
a cohesive design that enables more principal types to be inferred 
and more subtyping relations to be verified than in previous systems. 
In particular, our system is able to learn recursive types of function inputs
and outputs purely from the construction of recursive functions without 
any intrinsic recursive type specifications.
Additionally, our system can perform inductive reasoning to
verify than one recursive type subtypes another.
In the future, we will extend the recursive type reasoning capabilities
to handle recursive relational types, enabling the infinite paths
of recursive functions to be represented and reasoned with at the type level. 
With inductive reasoning over relational types, we will be
moving closer to a full fledged theorem proving system,
in which subtyping enables a form of proof automation,
by reusing the proof of one proposition as a proof for another. 
We may also explore different strategies for rewriting types 
to get as much expressivity as possible.


\newpage

\begin{thebibliography}{9}

% \bibitem{syngar}
% X. Wang, I, Dillig, R. Singh. Program synthesis using abstraction refinement. In POPL, 2017.

% \bibitem{love}
% M. Heizmann, J. Hoenicke, A. Podelski. Software model checking for people who love automata. In CAV, 2013. 

% \bibitem{synquid}
% N. Polikarpova, I. Kuraj, A. Solar-Lezama. Program synthesis from polymorphic refinement types. In PLDI, 2016.

\end{thebibliography}

\end{document}