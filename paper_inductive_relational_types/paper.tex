\documentclass[acmsmall]{acmart}
% \documentclass[letterpaper]{llncs}
% \usepackage[letterpaper, margin=1.5in]{geometry}


\usepackage{multicol}
\usepackage{mathpartir}
\usepackage{hyperref}
\usepackage{mathtools}
\usepackage{amsmath}
\usepackage{nccmath}
\usepackage{stmaryrd}
\usepackage{listings}
\usepackage[scaled]{beramono}
\usepackage[T1]{fontenc}
\usepackage[dvipsnames]{xcolor}

\usepackage{graphicx}
\graphicspath{ {./images/} }

\usepackage{url}

\makeatletter % allow us to mention @-commands
\def\arcr{\@arraycr}
\makeatother

\lstset{
    % identifierstyle=\color{violet},
    % textcolor=blue,
    % keywordstyle=\color{blue},
    % keywordstyle=\text,
    basicstyle=\ttfamily\small,
    % mathescape=true,
    % showspaces=false,
    % morekeywords={let, fix, in}
}

\theoremstyle{definition}
\newtheorem{theorem}{Theorem}[section]
\newtheorem{definition}{Definition}[section]
% \newtheorem{proof}{Proof}[section]


\title{Inductive Relational Types}
% \author{}
% \date{}

\begin{document}

% \newcommand{\obj}[1]{\texttt{\small #1}}
\newcommand{\obj}[1]{\texttt{\color{RoyalBlue} #1}}
\newcommand{\lab}[1]{\small \text{\color{Gray}\ [#1]}}
% \newcommand{\obj}[1]{\texttt{#1}}
% \newcommand{\obj}[1]{\textbf{\texttt{#1}}}
\newcommand{\entails}{\vdash}
\newcommand{\satisfies}{\vDash}
\newcommand{\given}{\dashv}
\newcommand{\with}{\ \diamond\ }
\newcommand{\notfree}{\ \#\ }
\newcommand{\consis}{\ \star}
\newcommand{\safe}{\ \checkmark}
\newcommand{\relational}{\ \Re}

\newcommand{\factorsinto}{\Vvdash}


\newcommand{\allsafe}{\ \Re\checkmark}



\newcommand{\foreign}{\varnothing}
\newcommand{\closed}{\bullet}
\newcommand{\local}{\blacktriangle}
\newcommand{\open}{\circ}


% \newcommand{\tl}{\textasciitilde{}}

\newcommand{\multi}[1]{\widebar{\ #1\ }}
\newcommand{\hastype}{:}
\newcommand{\pattype}{\ \lozenge\ }
\newcommand{\liftfun}{:}
 
\newcommand{\subtypes}{<:}
\newcommand{\supertypes}{:>}
\newcommand{\I}{\hspace{4mm}}
\newcommand{\Z}{.\hspace{4mm}}
\newcommand{\Alpha}{\mathrm{A}}
\newcommand{\Tau}{\mathrm{T}}
\newcommand{\B}[1]{\textbf{#1}}
\newcommand{\F}[1]{\text{#1}}
\newcommand{\bigand}{\bigwedge\nolimits}
\newcommand{\bigor}{\bigvee\nolimits}
\newcommand{\C}[1]{\color{teal} \rhd\ \emph{#1}}
\newcommand{\com}[1]{\color{Gray} \emph{#1}}
\newcommand{\D}[1]{\small \textsc{#1}}
% \newcommand{\fig}[1]{Fig. {\color{red} \fig{#1}}}
\newcommand{\FIG}[1]{Fig. {\color{red} \ref{#1}}}
\newcommand{\TODO}[1]{\noindent \B{\color{red} TODO: #1}}

\newcommand{\is}{\ ::=\ }
\newcommand{\sep}{\ \ |\ \ }
\newcommand{\nonterm}[1]{#1\ }
\newcommand{\contin}{|\ \ \ \ \ \ \ }
\newcommand{\case}{\B{case }}
\newcommand{\wrt}{\B{wrt }}
% \newcommand{\let}{\B{let }}
% \newcommand{\for}{\B{for }}
     


\newcommand{\tl}{\textasciitilde{}}
\newcommand{\fieldmap}{\obj{\tl>}}
\newcommand{\typdiff}{\obj{\textbackslash}}

\maketitle


\section{Introduction}

\section{Overview}

\section{Expression System}

\section{Type System}

\hfill
\begin{definition} 
  \label{def:relational_proof_typing}
  Rel Proof Typing (Loop)
  \hfill 
  \boxed{\Gamma \entails e \hastype \tau \given \Omega}
  \\
  \begin{mathpar}

    \inferrule {
      \Gamma \entails e \hastype \alpha_\nu\obj{->}\tau \given  \vec{\alpha}, \Delta 
      \\\\
      \forall \vec{\alpha}'\ \Delta' .\ 
        \vec{\alpha}', \llbracket \epsilon, \vec{\alpha}\ \vec{\alpha}' \entails \Delta', \alpha_l \obj{->} \alpha_r \rrbracket^+ \in \vec{\pi}_\nu
      \implies
      \tau \subtypes \alpha_l\obj{->}\alpha_r \given  \vec{\alpha}\ \vec{\alpha}', \Delta\ \Delta' 
      \\\\
      \exists \pi.\ \pi \in \vec{\pi}_\nu
      \\
      \alpha_\nu \downarrow \vec{\pi}_\nu
      \fallingdotseq 
      \alpha_\mu \uparrow \vec{\pi}_\mu
      \\
      \text{pack}^-(\text{FTV}(\Gamma)\ \vec{\alpha}\ \alpha_\mu \entails \vec{\pi}_\mu) = \tau_\mu
    } {
      \Gamma \entails \obj{loop(}e\obj{)} 
      \hastype 
      \obj{ALL[}\alpha_x \obj{]}
      \alpha_x
      \obj{->} 
      \obj{EXI[}\alpha_y 
          \obj{;} \alpha_x \obj{*} \alpha_y 
      \obj{<:LFP[} \alpha_\mu \obj{]}\tau_\mu
      \obj{]}\alpha_y
      \given \vec{\alpha}, \Delta 
    }

  \end{mathpar}
\end{definition}
\hfill

\hfill
\begin{definition} 
  \label{def:proof_subtyping_implication_rewriting}
  Rel Proof Subtyping (Implication Rewriting)
  \hfill
  \boxed{\tau \subtypes \tau \given \Omega}
  \\
  \begin{mathpar}
    \inferrule {
      \tau_l \subtypes
      \obj{ALL[}\alpha\obj{;} \alpha \obj{<:} \tau_\mu \obj{]} \alpha \obj{->} \tau_r
      \given \Omega
    } {
      \tau_l
      \subtypes 
      \obj{LFP[}\alpha_\mu\obj{]}\tau_\mu\obj{->}\tau_r
      \given \Omega 
    }
  \end{mathpar}
\end{definition}
\hfill


\hfill
\begin{definition} 
  \label{def:proof_subtyping_abstraction_introduction}
  Rel Proof Subtyping (Pattern Intro)
  \hfill 
  \boxed{\rho \subtypes \obj{LFP[}\alpha\obj{]}\tau \given \Omega}
  \\
  \begin{mathpar}
    \inferrule {
      \rho \subtypes \obj{LFP[}\alpha\obj{]}\tau \cong \rho' \subtypes \obj{LFP[}\alpha \obj{]} \tau'
      \\
      \Delta \entails \rho' \subtypes \tau_n \sim 
      \\\\
      (\vec{\alpha}, \Delta) \preceq \Omega
      % \\
      % \exists \alpha .\ \alpha \in \text{FTV}(\rho) \land \alpha \in \vec{\alpha} 
      \\
      \tau_n \subtypes \obj{LFP[}\alpha \obj{]} \tau' \entails \Omega
    } {
      \rho \subtypes \obj{LFP[}\alpha\obj{]}\tau
      \given \Omega 
    }

    \inferrule {
      \entails \rho \subtypes \obj{LFP[}\alpha\obj{]}\tau \consis 
      \\
      \forall \alpha_\closed .\ 
      \alpha_\closed \in \text{FTV}(\rho) \land \alpha_\closed \in \vec{\alpha}_\closed
      \implies
      (\vec{\alpha}_\closed, \Delta) \entails \alpha_\closed \wr \rho \subtypes \obj{LFP[}\alpha\obj{]}\tau \safe
      \\\\
      \exists \Delta_i .\ \llbracket \epsilon, \vec{\alpha} \entails \Delta, \rho\rrbracket^+ = \Delta_i, \rho'
      \\
      (\vec{\alpha}, \Delta) \preceq (\vec{\alpha}', \Delta')
      \\
      \rho' \subtypes \obj{LFP[}\alpha\obj{]}\tau
      \given \vec{\alpha}', \Delta'
    } {
      \rho \subtypes \obj{LFP[}\alpha\obj{]}\tau
      \given \vec{\alpha}', \Delta'\ \obj{;} \rho \obj{<:} \obj{LFP[}\alpha\obj{]}\tau
    }
  \end{mathpar}
\end{definition}
\hfill

\hfill
\begin{definition} 
  \label{def:rel_closed_variable_safety_negative}
  Rel Closed Variable Safety (Negative)
  \hfill
  \boxed{(\vec{\alpha}_\closed, \Delta) \entails \alpha_\closed \subtypes \tau \safe}
  \\
  \begin{mathpar}
    \inferrule {
      \nexists \alpha .\ \tau_l = \alpha \land \alpha \notin \vec{\alpha}_\closed
      \\
      \exists \Omega .\ 
      (\vec{\alpha}_\closed, \Delta) \preceq \Omega \land
      \tau_l \subtypes \tau_r \given \Omega 
    } {
      (\vec{\alpha}_\closed, \Delta\ \obj{;} \alpha_\closed \obj{<:} \tau_l) \entails \alpha_\closed \subtypes \tau_r \safe
    }

    \inferrule {
      (\vec{\alpha}_\closed, \Delta) \entails \alpha_\closed \subtypes \tau \safe
    } {
      (\vec{\alpha}_\closed, \Delta\ \obj{;} \tau_l \obj{<:} \tau_r) \entails \alpha_\closed \subtypes \tau \safe
    }

    \inferrule {
      \alpha_\closed \in \text{FTV}(\rho)
      \\
      \rho \subtypes \tau \factorsinto \alpha_\closed \subtypes \tau' 
      \\
      (\vec{\alpha}_\closed, \Delta\ \obj{;} \alpha_\closed \obj{<:} \tau') \entails \alpha_\closed \subtypes \tau'' \safe
    } {
      (\vec{\alpha}_\closed, \Delta\ \obj{;} \rho \obj{<:} \tau) \entails \alpha_\closed \subtypes \tau'' \safe
    }
  \end{mathpar}
\end{definition}
\hfill

\hfill
\begin{definition} 
  \label{def:relational_key_safety_negative}
  Relational Key Safety (Negative)
  \hfill
  \boxed{(\vec{\alpha}_\closed, \Delta) \entails \alpha_\closed \wr \rho \subtypes \tau \safe}
  \\
  \begin{mathpar}
    \inferrule {
      \nexists \alpha .\ \tau_l = \alpha \land \alpha \notin \vec{\alpha}_\closed
      \\
      \forall \tau_r .\
      \rho \subtypes \tau \factorsinto \alpha_\closed \subtypes \tau_r
      \implies
      \exists \Omega .\  
      (\vec{\alpha}_\closed, \Delta) \preceq \Omega \land
      \tau_l \subtypes \tau_r \given \Omega 
    } {
      (\vec{\alpha}_\closed, \Delta\ \obj{;} \alpha_\closed \obj{<:} \tau_l) 
      \entails 
      \alpha_\closed  \wr \rho \subtypes \tau \safe
    }

    \inferrule {
      (\vec{\alpha}_\closed, \Delta) \entails \alpha_\closed \wr \rho \subtypes \tau \safe
    } {
      (\vec{\alpha}_\closed, \Delta\ \obj{;} \tau_l \obj{<:} \tau_r) 
      \entails
      \alpha_\closed  \wr \rho \subtypes \tau \safe
    }

    \inferrule {
      \alpha_\closed \in \text{FTV}(\rho)
      \\
      \rho \subtypes \tau \factorsinto \alpha_\closed \subtypes \tau' 
      \\\\
      (\vec{\alpha}_\closed, \Delta\ \obj{;} \alpha_\closed \obj{<:} \tau') \entails \alpha_\closed \wr \rho' \subtypes \tau'' \safe
    } {
      (\vec{\alpha}_\closed, \Delta\ \obj{;} \rho \obj{<:} \tau) \entails \alpha_\closed \wr \rho' \subtypes \tau'' \safe
    }
  \end{mathpar}
\end{definition}
\hfill


\hfill
\begin{definition} 
  \label{def:relational_constraint_factorization}
  Relational Constraint Factorization 
  \hfill
  \boxed{\rho \subtypes \tau \factorsinto \alpha \subtypes \tau'}
  \\
  \begin{mathpar}
    \inferrule {
      \TODO{...}
    } {
      \rho \subtypes \tau \factorsinto \alpha \subtypes \tau' 
    }
  \end{mathpar}
\end{definition}
\hfill


\hfill
\begin{definition} 
  \label{def:proof_subtyping_abstraction_introduction}
  Rel Proof Subtyping (Abstraction Intro)
  \hfill
  \boxed{\tau \subtypes \phi \given \Omega}
  \\
  \begin{mathpar}
    \inferrule {
      \rho \subtypes \tau_{l} \given \Omega 
    } {
      \rho \subtypes \tau_{l}\obj{|}\tau_{r} \given \Omega 
    }

    \inferrule {
      \rho \subtypes \tau_{r} \given \Omega 
    } {
      \rho \subtypes \tau_{l}\obj{|}\tau_{r} \given \Omega 
    }
  \end{mathpar}
\end{definition}
\hfill

\hfill
\begin{definition}
  \label{def:rel_interpretation}
  Rel Interpretation  (Negative)
  \hfill
  \boxed{\llbracket \Delta, \alpha \rrbracket^- = (\Delta, \tau)}
  \\
  \begin{mathpar}
    \inferrule {
      \neg(\Delta \entails \alpha \relational)
      \\
    } {
      \llbracket \epsilon, \alpha \rrbracket^- = (\epsilon, \obj{TOP})
    }

    \inferrule {
      \neg(\Delta \entails \alpha \relational)
      \\
      \llbracket \Delta, \alpha \rrbracket^- = (\Delta', \tau')
    } {
      \llbracket \Delta\ \obj{;}\alpha  \obj{<:} \tau, \alpha \rrbracket^- = (\Delta', \tau' \obj{\&} \tau)
    }

    \inferrule {
      \alpha \neq \tau_l
      \\
      \llbracket \Delta, \alpha \rrbracket^- = (\Delta', \tau)
    } {
      \llbracket \Delta\ \obj{;}\tau_l \obj{<:} \tau_r, \alpha \rrbracket^- = (\Delta'\ \obj{;}\tau_l \obj{<:} \tau_r, \tau)
    }

    \inferrule {
      \Delta \entails \alpha \relational
    } {
      \llbracket \Delta, \alpha \rrbracket^- = (\Delta, \alpha)
    }
  \end{mathpar}
\end{definition}
\hfill

\hfill
\begin{definition}
  \label{def:relational_constraint_key}
  Relational Constraint Key
  \hfill
  \boxed{\Delta \entails \alpha \relational}
  \\
  \begin{mathpar}
    \inferrule {
      \obj{;} \tau_l \obj{<:} \tau_r \in \Delta 
      \\
      \nexists \alpha' .\ \tau_l = \alpha'
      \\
      \alpha \in \text{FTV}(\tau_l)
    } {
      \Delta \entails \alpha \relational
    }
  \end{mathpar}
\end{definition}
\hfill


\hfill
\begin{definition}
  \label{def:decomposable}
  Decomposable 
  \\\\
  \boxed{\Omega \entails \tau \circlearrowleft \tau}
  \begin{mathpar}
    \inferrule {
      \TODO{...}
    } {
      % \text{decomposable}(\Delta_0, \tau_l, \obj{LFP[} \alpha \obj{]} \tau_r)
      \Omega \entails \tau \circlearrowleft \tau
    }
  \end{mathpar}
\end{definition}
\hfill

\hfill
\begin{definition}\boxed{\tau \subtypes \tau \cong \tau \subtypes \tau}\ Normal Constraint Congruence 
  \label{def:normal_constraint_congruence}
  \begin{mathpar}
    \inferrule {
      \TODO{...}
    } {
      \tau \subtypes \tau \cong \tau \subtypes \tau
    }
  \end{mathpar}
\end{definition}
\hfill

\hfill
\begin{definition}\boxed{\Delta \entails \tau' \subtypes \tau \sim}\ Normal Constraint Entailment  
  \label{def:normal_constraint_entailment}
  \begin{mathpar}
    \inferrule {
      \TODO{...}
    } {
      \Delta \entails \tau' \subtypes \tau \sim
    }
  \end{mathpar}
\end{definition}
\hfill

\section{Correctness}

\section{Experiments}

\section{Related work}

\section{Future Work}

\section{Appendix}

\end{document}