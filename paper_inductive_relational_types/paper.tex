\documentclass[acmsmall]{acmart}
% \documentclass[letterpaper]{llncs}
% \usepackage[letterpaper, margin=1.5in]{geometry}


\usepackage{multicol}
\usepackage{mathpartir}
\usepackage{hyperref}
\usepackage{mathtools}
\usepackage{amsmath}
\usepackage{nccmath}
\usepackage{stmaryrd}
\usepackage{listings}
\usepackage[scaled]{beramono}
\usepackage[T1]{fontenc}
\usepackage[dvipsnames]{xcolor}

\usepackage{graphicx}
\graphicspath{ {./images/} }

\usepackage{url}

\makeatletter % allow us to mention @-commands
\def\arcr{\@arraycr}
\makeatother

\lstset{
    % identifierstyle=\color{violet},
    % textcolor=blue,
    % keywordstyle=\color{blue},
    % keywordstyle=\text,
    basicstyle=\ttfamily\small,
    % mathescape=true,
    % showspaces=false,
    % morekeywords={let, fix, in}
}

\theoremstyle{definition}
\newtheorem{theorem}{Theorem}[section]
\newtheorem{definition}{Definition}[section]
% \newtheorem{proof}{Proof}[section]


\title{Inductive Relational Types}
% \author{}
% \date{}

\begin{document}

\newcommand{\J}[1]{\texttt{\color{RoyalBlue} #1}}

\newcommand{\lab}[1]{\small \text{\color{Gray}\ [#1]}}
\newcommand{\entails}{\vdash}
\newcommand{\satisfies}{\vDash}
\newcommand{\given}{\dashv}
\newcommand{\with}{\ \diamond\ }
\newcommand{\notfree}{\ \#\ }
\newcommand{\consis}{\ \star}
\newcommand{\safe}{\ \checkmark}
\newcommand{\relational}{\ \Re}

\newcommand{\factorsinto}{\Vvdash}


\newcommand{\allsafe}{\ \Re\checkmark}



\newcommand{\foreign}{\varnothing}
\newcommand{\closed}{\bullet}
\newcommand{\local}{\blacktriangle}
\newcommand{\open}{\circ}


% \newcommand{\tl}{\textasciitilde{}}

\newcommand{\multi}[1]{\widebar{\ #1\ }}
\newcommand{\hastype}{:}
\newcommand{\pattype}{\ \lozenge\ }
\newcommand{\liftfun}{:}
 
\newcommand{\subtypes}{<:}
\newcommand{\supertypes}{:>}
\newcommand{\I}{\hspace{4mm}}
\newcommand{\Z}{.\hspace{4mm}}
\newcommand{\Alpha}{\mathrm{A}}
\newcommand{\Tau}{\mathrm{T}}
\newcommand{\B}[1]{\textbf{#1}}
\newcommand{\F}[1]{\text{#1}}
\newcommand{\bigand}{\bigwedge\nolimits}
\newcommand{\bigor}{\bigvee\nolimits}
\newcommand{\C}[1]{\color{teal} \rhd\ \emph{#1}}
\newcommand{\com}[1]{\color{Gray} \emph{#1}}
\newcommand{\D}[1]{\small \textsc{#1}}
% \newcommand{\fig}[1]{Fig. {\color{red} \fig{#1}}}
\newcommand{\FIG}[1]{Fig. {\color{red} \ref{#1}}}
\newcommand{\TODO}[1]{\noindent \B{\color{red} TODO: #1}}

\newcommand{\is}{\ ::=\ }
\newcommand{\sep}{\ \ |\ \ }
\newcommand{\nonterm}[1]{#1\ }
\newcommand{\contin}{|\ \ \ \ \ \ \ }
\newcommand{\case}{\B{case }}
\newcommand{\wrt}{\B{wrt }}
% \newcommand{\let}{\B{let }}
% \newcommand{\for}{\B{for }}
     


\newcommand{\tl}{\textasciitilde{}}
\newcommand{\fieldmap}{\J{\tl>}}
\newcommand{\typdiff}{\J{\textbackslash}}

\maketitle


\section{Introduction}

\section{Overview}

\paragraph{Infinite Path Preservation} 
Consider the function \J{repeat} that takes a natural number and returns a list of whose length is that number. 

\[
  \begin{array}[t]{@{} l}
      \J{let repeat = ? x => loop(? self =>}
      \\
      \I \J{? <zero> @ => <nil> @}
      \\
      \I \J{? <succ> n => <cons>(x,self(n))}
      \\
      \J{)}
  \end{array}
\]


Without specifying any requirements besides the function definition, type inference lifts 
the function into the definitional property as a type. 
To construct the type, type inference constructs a relation between nats and lists.
The type of \J{repeat} depends on a least fixed point relation between nats and lists 
(parametrically named here for readability).

\[
  \text{natList}(\alpha) = \begin{array}[t]{@{} l}
      \J{LFP[R]( BOT} 
      \\
      \I \J{| (<zero> @)*(<nil> @)}
      \\
      \I \J{| (EXI[N L ; N*L <: R](<succ> N)*(<cons> (}\alpha\J{*L))}
      \\
      \J{)} 
  \end{array}
\]

Using the \text{natList} relation, type inference then lifts the function repeat into its precise type form.

\[
  \J{ALL[T] T -> ALL[X] X -> EXI[Y ; X*Y <: }\text{natList}(\J{T}) \J{] Y}
\]

It may be worth noting that there could be semantically equivalent recursive type in terms of intersections instead of unions. 

\TODO{forward reference to correctness/model semantics}

\[
  \begin{array}[t]{@{} l}
      \J{ALL[T] GFP[R]( TOP} 
      \\
      \I \J{\& (<zero> @)->(<nil> @)}
      \\
      \I \J{\& (ALL[N L ; R <: N->L](<succ> N)->(<cons> T*L)}
      \\
      \J{)} 
  \end{array}
\]


Type inference reasons in terms least fixed points, but the greatest fixed point form could
be handled with syntactic sugar and rewriting.  

Using the precise type form, type inference can leverage solving and checking subtyping constraints 
to reason in a number ways: it can reason forward from inputs to outputs (just like the runtime semantics), 
reason backwards from outputs to inputs (like Prolog), and check against weaker specifications. 


\paragraph{Infinite Paths Selection.} Consider the application \J{repeat(<succ> <succ> <zero> @)(x)} where \J{x}
has type \J{T}. Type inference
constructs a singleton type, mirroring the results achieved by simply running the program.  
\[
  \J{<cons> T * <cons> T * <nil> @}
\]

Now suppose we have a function \J{foo} whose input
type is inferred to be an empty list or a singleton list, \J{<nil> @ |  <cons> T * <nil> @}.
Given the application \J{foo(repeat(n)(x))} where \J{x} has type \J{T}, type inference can reason backwards to learn
that the type of \J{n} must be either zero or one. 
\[
  \J{<zero> @ | <succ> <zero> @}
\]



\paragraph{Infinite Path Factorization.} 
Now suppose we have a function \J{woo} whose input
type is inferred to be a list over elements of type \J{T}.
\[
  \begin{array}[t]{@{} l}
      \J{LFP[R]( BOT}
      \\
      \I \J{| <nil> @ }
      \\
      \I \J{| <cons> T*R}
      \\
      \J{)}
  \end{array}
\]

Given the application \J{woo(repeat(n)(x))} where \J{x} has type \J{T}, type 
inference discovers that the argument type depends on the relation \text{natList}(\J{T}), 
and the relation can be factored into a weaker cross product of nats and lists. 
Therefore, the argument meets the requirements
of \J{woo} and the type of \J{n} must be the natural numbers.

\[
  \begin{array}[t]{@{} l}
      \J{LFP[R]( BOT} 
      \\
      \I \J{| <zero> @}
      \\
      \I \J{| <succ> R}
      \\
      \J{)} 
  \end{array}
\]



\paragraph{Relational Induction.} 
\TODO{...}
\paragraph{Path Sensitivity.} Consider the function \J{max} that chooses the maximum of two natural numbers. 



\[
  \begin{array}[t]{@{} l}
      \J{let lessOrEq = loop(? self =>}
      \\
      \I \J{? (<zero> @),y => <true> @}
      \\
      \I \J{? (<succ> x),(<succ> y) => self(x,y)}
      \\
      \I \J{? (<succ> x),(<zero> @) => <false> @}
      \\
      \J{) in}
      \\
      \J{let max = ? (x,y) => }
      \\
      \I \J{if lessOrEq(x,y) then y else x}
  \end{array}
\]


The function \J{max} must satisfy the property that the result is greater or equal to each of the inputs. 
Type inference must learn constraints on the inputs to \J{max}: \J{x} and \J{y}, which 
depends on the output of \J{lessOrEq(x,y)}. The application \J{lessOrEq(x,y)} can evaluate to either \J{<true>@}
or \J{<false>@}, which result from different paths taken in \J{lessOrEq}. 
Type inference considers both cases and maintains the learned constraints
exist in different possible worlds, since they are learned from different paths. Finally, type inference
connects the inputs to the outputs by considering the two possible paths of the \J{if-then-else}.
It first lifts the function \J{lessOrEq} into a relational type. For readability, we name the relational type \text{LED} (as in "less than or equal decision"). 

\[
  \text{LED} = 
  \begin{array}[t]{@{} l}
      \J{LFP[R] ( BOT}
      \\
      \I \J{| (EXI [Y] ((<zero> @)*Y)*(<true> @))}
      \\
      \I \J{| (EXI [X Y D ; ((X*Y)*Z) <: R] ((<succ> X)*(<succ> Y))*D)}
      \\
      \I \J{| (EXI [X] ((<succ> X)*(<zero> @))*(<false> @))}
      \\
      \J{)}
  \end{array}
\]

using the \text{LED} relation, type inference combines the constraints learned
for each possible world and combines them into a function type with multiple paths. 

\[
  \begin{array}[t]{@{} l}
    \J{TOP}
    \\
    \J{\& (EXI [D ; D <: (<true> @)]}
    \\
    \I \J{(ALL [X Y Z ; Y <: Z ; ((X*Y)*D) <:}\text{LED}\J{] (X,Y) -> Z))}
    \\
    \J{\& (EXI [D ; D <: (<false> @)]}
    \\
    \I \J{(ALL [X Y Z ; X <: Z ; ((X*Y)*D) <: }\text{LED}\J{] (X,Y) -> Z))}
  \end{array}
\]

We could simplify the type by eliminating simple constraints without loss of safety or precision:

\[
  \begin{array}[t]{@{} l}
    \J{TOP}
    \\
    \J{\& (ALL [X Y ; ((X*Y)*(<true> @)) <: }\text{LED}\J{] (X,Y) -> Y)}
    \\
    \J{\& (ALL [X Y ; ((X*Y)*(<false> @)) <: }\text{LED}\J{] (X,Y) -> X)}
  \end{array}
\]

However, we have not included this rewriting in the semantics or implementation. 


\section{Expression System}

\section{Type System}

\hfill
\begin{definition} 
  \label{def:relational_proof_typing}
  Rel Proof Typing (Loop)
  \hfill 
  \boxed{\Gamma \entails e \hastype \tau \given \Omega}
  \\
  \begin{mathpar}

    \inferrule {
      \Gamma \entails e \hastype \alpha_\nu\J{->}\tau \given  \vec{\alpha}, \Delta 
      \\\\
      \forall \vec{\alpha}'\ \Delta' .\ 
        \vec{\alpha}', \llbracket \epsilon, \vec{\alpha}\ \vec{\alpha}' \entails \Delta', \alpha_l \J{->} \alpha_r \rrbracket^+ \in \vec{\pi}_\nu
      \implies
      \tau \subtypes \alpha_l\J{->}\alpha_r \given  \vec{\alpha}\ \vec{\alpha}', \Delta\ \Delta' 
      \\\\
      \exists \pi.\ \pi \in \vec{\pi}_\nu
      \\
      \alpha_\nu \downarrow \vec{\pi}_\nu
      \fallingdotseq 
      \alpha_\mu \uparrow \vec{\pi}_\mu
      \\
      \text{pack}^-(\text{FTV}(\Gamma)\ \vec{\alpha}\ \alpha_\mu \entails \vec{\pi}_\mu) = \tau_\mu
    } {
      \Gamma \entails \J{loop(}e\J{)} 
      \hastype 
      \J{ALL[}\alpha_x \J{]}
      \alpha_x
      \J{->} 
      \J{EXI[}\alpha_y 
          \J{;} \alpha_x \J{*} \alpha_y 
      \J{<:LFP[} \alpha_\mu \J{]}\tau_\mu
      \J{]}\alpha_y
      \given \vec{\alpha}, \Delta 
    }

  \end{mathpar}
\end{definition}
\hfill

\hfill
\begin{definition} 
  \label{def:proof_subtyping_implication_rewriting}
  Rel Proof Subtyping (Implication Rewriting)
  \hfill
  \boxed{\tau \subtypes \tau \given \Omega}
  \\
  \begin{mathpar}
    \inferrule {
      \tau_l \subtypes
      \J{ALL[}\alpha\J{;} \alpha \J{<:} \tau_\mu \J{]} \alpha \J{->} \tau_r
      \given \Omega
    } {
      \tau_l
      \subtypes 
      \J{LFP[}\alpha_\mu\J{]}\tau_\mu\J{->}\tau_r
      \given \Omega 
    }
  \end{mathpar}
\end{definition}
\hfill


\hfill
\begin{definition} 
  \label{def:proof_subtyping_abstraction_introduction}
  Rel Proof Subtyping (Pattern Intro)
  \hfill 
  \boxed{\rho \subtypes \J{LFP[}\alpha\J{]}\tau \given \Omega}
  \\
  \begin{mathpar}
    \inferrule {
      \rho \subtypes \J{LFP[}\alpha\J{]}\tau \cong \rho' \subtypes \J{LFP[}\alpha \J{]} \tau'
      \\
      \Delta \entails \rho' \subtypes \tau_n \sim 
      \\\\
      (\vec{\alpha}, \Delta) \preceq \Omega
      % \\
      % \exists \alpha .\ \alpha \in \text{FTV}(\rho) \land \alpha \in \vec{\alpha} 
      \\
      \tau_n \subtypes \J{LFP[}\alpha \J{]} \tau' \entails \Omega
    } {
      \rho \subtypes \J{LFP[}\alpha\J{]}\tau
      \given \Omega 
    }

    \inferrule {
      \entails \rho \subtypes \J{LFP[}\alpha\J{]}\tau \consis 
      \\
      \forall \alpha_\closed .\ 
      \alpha_\closed \in \text{FTV}(\rho) \land \alpha_\closed \in \vec{\alpha}_\closed
      \implies
      (\vec{\alpha}_\closed, \Delta) \entails \alpha_\closed \wr \rho \subtypes \J{LFP[}\alpha\J{]}\tau \safe
      \\\\
      \exists \Delta_i .\ \llbracket \epsilon, \vec{\alpha} \entails \Delta, \rho\rrbracket^+ = \Delta_i, \rho'
      \\
      (\vec{\alpha}, \Delta) \preceq (\vec{\alpha}', \Delta')
      \\
      \rho' \subtypes \J{LFP[}\alpha\J{]}\tau
      \given \vec{\alpha}', \Delta'
    } {
      \rho \subtypes \J{LFP[}\alpha\J{]}\tau
      \given \vec{\alpha}', \Delta'\ \J{;} \rho \J{<:} \J{LFP[}\alpha\J{]}\tau
    }
  \end{mathpar}
\end{definition}
\hfill

\hfill
\begin{definition} 
  \label{def:rel_closed_variable_safety_negative}
  Rel Closed Variable Safety (Negative)
  \hfill
  \boxed{(\vec{\alpha}_\closed, \Delta) \entails \alpha_\closed \subtypes \tau \safe}
  \\
  \begin{mathpar}
    \inferrule {
      \nexists \alpha .\ \tau_l = \alpha \land \alpha \notin \vec{\alpha}_\closed
      \\
      \exists \Omega .\ 
      (\vec{\alpha}_\closed, \Delta) \preceq \Omega \land
      \tau_l \subtypes \tau_r \given \Omega 
    } {
      (\vec{\alpha}_\closed, \Delta\ \J{;} \alpha_\closed \J{<:} \tau_l) \entails \alpha_\closed \subtypes \tau_r \safe
    }

    \inferrule {
      (\vec{\alpha}_\closed, \Delta) \entails \alpha_\closed \subtypes \tau \safe
    } {
      (\vec{\alpha}_\closed, \Delta\ \J{;} \tau_l \J{<:} \tau_r) \entails \alpha_\closed \subtypes \tau \safe
    }

    \inferrule {
      \alpha_\closed \in \text{FTV}(\rho)
      \\
      \rho \subtypes \tau \factorsinto \alpha_\closed \subtypes \tau' 
      \\
      (\vec{\alpha}_\closed, \Delta\ \J{;} \alpha_\closed \J{<:} \tau') \entails \alpha_\closed \subtypes \tau'' \safe
    } {
      (\vec{\alpha}_\closed, \Delta\ \J{;} \rho \J{<:} \tau) \entails \alpha_\closed \subtypes \tau'' \safe
    }
  \end{mathpar}
\end{definition}
\hfill

\hfill
\begin{definition} 
  \label{def:relational_key_safety_negative}
  Relational Key Safety (Negative)
  \hfill
  \boxed{(\vec{\alpha}_\closed, \Delta) \entails \alpha_\closed \wr \rho \subtypes \tau \safe}
  \\
  \begin{mathpar}
    \inferrule {
      \nexists \alpha .\ \tau_l = \alpha \land \alpha \notin \vec{\alpha}_\closed
      \\
      \forall \tau_r .\
      \rho \subtypes \tau \factorsinto \alpha_\closed \subtypes \tau_r
      \implies
      \exists \Omega .\  
      (\vec{\alpha}_\closed, \Delta) \preceq \Omega \land
      \tau_l \subtypes \tau_r \given \Omega 
    } {
      (\vec{\alpha}_\closed, \Delta\ \J{;} \alpha_\closed \J{<:} \tau_l) 
      \entails 
      \alpha_\closed  \wr \rho \subtypes \tau \safe
    }

    \inferrule {
      (\vec{\alpha}_\closed, \Delta) \entails \alpha_\closed \wr \rho \subtypes \tau \safe
    } {
      (\vec{\alpha}_\closed, \Delta\ \J{;} \tau_l \J{<:} \tau_r) 
      \entails
      \alpha_\closed  \wr \rho \subtypes \tau \safe
    }

    \inferrule {
      \alpha_\closed \in \text{FTV}(\rho)
      \\
      \rho \subtypes \tau \factorsinto \alpha_\closed \subtypes \tau' 
      \\\\
      (\vec{\alpha}_\closed, \Delta\ \J{;} \alpha_\closed \J{<:} \tau') \entails \alpha_\closed \wr \rho' \subtypes \tau'' \safe
    } {
      (\vec{\alpha}_\closed, \Delta\ \J{;} \rho \J{<:} \tau) \entails \alpha_\closed \wr \rho' \subtypes \tau'' \safe
    }
  \end{mathpar}
\end{definition}
\hfill


\hfill
\begin{definition} 
  \label{def:relational_constraint_factorization}
  Relational Constraint Factorization 
  \hfill
  \boxed{\rho \subtypes \tau \factorsinto \alpha \subtypes \tau'}
  \\
  \begin{mathpar}
    \inferrule {
      \TODO{...}
    } {
      \rho \subtypes \tau \factorsinto \alpha \subtypes \tau' 
    }
  \end{mathpar}
\end{definition}
\hfill


\hfill
\begin{definition} 
  \label{def:proof_subtyping_abstraction_introduction}
  Rel Proof Subtyping (Abstraction Intro)
  \hfill
  \boxed{\tau \subtypes \phi \given \Omega}
  \\
  \begin{mathpar}
    \inferrule {
      \rho \subtypes \tau_{l} \given \Omega 
    } {
      \rho \subtypes \tau_{l}\J{|}\tau_{r} \given \Omega 
    }

    \inferrule {
      \rho \subtypes \tau_{r} \given \Omega 
    } {
      \rho \subtypes \tau_{l}\J{|}\tau_{r} \given \Omega 
    }
  \end{mathpar}
\end{definition}
\hfill

\hfill
\begin{definition}
  \label{def:rel_interpretation}
  Rel Interpretation  (Negative)
  \hfill
  \boxed{\llbracket \Delta, \alpha \rrbracket^- = (\Delta, \tau)}
  \\
  \begin{mathpar}
    \inferrule {
      \neg(\Delta \entails \alpha \relational)
      \\
    } {
      \llbracket \epsilon, \alpha \rrbracket^- = (\epsilon, \J{TOP})
    }

    \inferrule {
      \neg(\Delta \entails \alpha \relational)
      \\
      \llbracket \Delta, \alpha \rrbracket^- = (\Delta', \tau')
    } {
      \llbracket \Delta\ \J{;}\alpha  \J{<:} \tau, \alpha \rrbracket^- = (\Delta', \tau' \J{\&} \tau)
    }

    \inferrule {
      \alpha \neq \tau_l
      \\
      \llbracket \Delta, \alpha \rrbracket^- = (\Delta', \tau)
    } {
      \llbracket \Delta\ \J{;}\tau_l \J{<:} \tau_r, \alpha \rrbracket^- = (\Delta'\ \J{;}\tau_l \J{<:} \tau_r, \tau)
    }

    \inferrule {
      \Delta \entails \alpha \relational
    } {
      \llbracket \Delta, \alpha \rrbracket^- = (\Delta, \alpha)
    }
  \end{mathpar}
\end{definition}
\hfill

\hfill
\begin{definition}
  \label{def:relational_constraint_key}
  Relational Constraint Key
  \hfill
  \boxed{\Delta \entails \alpha \relational}
  \\
  \begin{mathpar}
    \inferrule {
      \J{;} \tau_l \J{<:} \tau_r \in \Delta 
      \\
      \nexists \alpha' .\ \tau_l = \alpha'
      \\
      \alpha \in \text{FTV}(\tau_l)
    } {
      \Delta \entails \alpha \relational
    }
  \end{mathpar}
\end{definition}
\hfill


\hfill
\begin{definition}
  \label{def:decomposable}
  Decomposable 
  \\\\
  \boxed{\Omega \entails \tau \circlearrowleft \tau}
  \begin{mathpar}
    \inferrule {
      \TODO{...}
    } {
      % \text{decomposable}(\Delta_0, \tau_l, \J{LFP[} \alpha \J{]} \tau_r)
      \Omega \entails \tau \circlearrowleft \tau
    }
  \end{mathpar}
\end{definition}
\hfill

\hfill
\begin{definition}\boxed{\tau \subtypes \tau \cong \tau \subtypes \tau}\ Normal Constraint Congruence 
  \label{def:normal_constraint_congruence}
  \begin{mathpar}
    \inferrule {
      \TODO{...}
    } {
      \tau \subtypes \tau \cong \tau \subtypes \tau
    }
  \end{mathpar}
\end{definition}
\hfill

\hfill
\begin{definition}\boxed{\Delta \entails \tau' \subtypes \tau \sim}\ Normal Constraint Entailment  
  \label{def:normal_constraint_entailment}
  \begin{mathpar}
    \inferrule {
      \TODO{...}
    } {
      \Delta \entails \tau' \subtypes \tau \sim
    }
  \end{mathpar}
\end{definition}
\hfill

\section{Correctness}

\section{Experiments}

\section{Related work}

\section{Future Work}

\section{Appendix}

\end{document}