\documentclass{beamer}

\usepackage{mathpartir}
\usepackage{hyperref}
\usepackage{mathtools}
\usepackage{listings}
\usepackage[utf8]{inputenc}


\lstset{
    % identifierstyle=\color{purple},
    % textcolor=blue,
    % keywordstyle=\color{blue},
    keywordstyle=\text,
    basicstyle=\ttfamily,
    mathescape=true,
    showspaces=false,
    % morekeywords={if, then, else, fn}
}

% \title{Subtyping Constraints and Type Inference}
% \author{Thomas Logan}
% % \date{18 November 2022}

\begin{document}

% \begin{frame}
%   \titlepage
% \end{frame}

\begin{frame}
  \frametitle{When Subtyping Constraints Liberate}

  \begin{itemize}
  \item LIONEL PARREAUX, HKUST, Hong Kong, China
  \item ALEKSANDER BORUCH-GRUSZECKI, EPFL, Switzerland
  \item ANDONG FAN, HKUST, Hong Kong, China
  \item CHUN YIN CHAU, HKUST, Hong Kong, China
  \end{itemize}
\end{frame}

\begin{frame}[fragile]
  \frametitle{Impoverished Type Inference}

  \begin{lstlisting}
foo f = (f 123, f True)
  \end{lstlisting}

\end{frame}


\begin{frame}[fragile]
  \frametitle{Satisfcatory Typing 1}

  \begin{itemize}
  \item should allow
  \begin{lstlisting}
foo (fn x => x)
  \end{lstlisting}
  \item where
  \begin{lstlisting}
(fn x => x) : ALL a . a -> a
  \end{lstlisting}
  \end{itemize}

\end{frame}


\begin{frame}[fragile]
  \frametitle{Satisfcatory Typing 2}

  \begin{itemize}
  \item should allow
  \begin{lstlisting}
foo (fn x => some x)
  \end{lstlisting}
  \item where
  \begin{lstlisting}
(fn x => some x) : ALL a . a -> Option a
  \end{lstlisting}
  \end{itemize}

\end{frame}

\begin{frame}[fragile]
  \frametitle{Liberation by Intersection}

  \begin{itemize}
  \item repeated application
  \begin{lstlisting}
foo f = (f 123, f True)
  \end{lstlisting}
  \item suggests intersection in parameter type  
  \begin{lstlisting}
foo : ALL a b . 
      ((Int -> a) & (Bool -> b)) -> (a,b)
  \end{lstlisting}
  \end{itemize}
\end{frame}


\begin{frame}[fragile]
  \frametitle{Instantiation as Subtyping}

  \begin{itemize}
  \item application 
  \begin{lstlisting}
foo (fn x => some x)
  \end{lstlisting}
  \item generates subtyping constraint to be checked or solved 
  \begin{lstlisting}
(ALL a . a -> Option a) 
         <: 
((Int -> c) & (Bool -> d))
  \end{lstlisting}
  \end{itemize}
\end{frame}


\begin{frame}[fragile]
  \frametitle{Constrained Variable as Intersection}
  \begin{itemize}
  \item intersection in parameter type 
  \begin{lstlisting}
ALL a b . 
  ((Int -> a) & (Bool -> b)) -> (a,b)
  \end{lstlisting}
  \item is the weakest interpretation of the parameter type in 
  \begin{lstlisting}
ALL a b c 
  {c <: Int -> a, c <: Bool -> b} . 
  c -> (a,b)
  \end{lstlisting}
  \end{itemize}
\end{frame}


\begin{frame}[fragile]
  \frametitle{Liberation by Union}

  \begin{itemize}
  \item branching  
  \begin{lstlisting}
bar f x = if (f x) then f else (fn x => x)
  \end{lstlisting}

  \item suggests union in return type 
  \begin{lstlisting}
bar : ALL a b . 
      (a & (b -> Bool)) -> 
      b -> (a | (ALL d . d -> d))
  \end{lstlisting}
  \end{itemize}
\end{frame}


\begin{frame}[fragile]
  \frametitle{Constrained Variable as Union}

  \begin{itemize}
  \item union in return type 
  \begin{lstlisting}
bar : ALL a b . 
      (a & (b -> Bool)) -> 
      b -> (a | (ALL d . d -> d))
  \end{lstlisting}
  \item is the strongest interpretation of the return type in 
  \begin{lstlisting}
bar : ALL a b c {
        a <: b -> Bool, 
        a <: c, 
        (ALL d . d -> d) <: c
      } . a -> b -> c
  \end{lstlisting}
  \end{itemize}
\end{frame}


\begin{frame}[fragile]
  \frametitle{Restricted User Annotations}

  \begin{itemize}
  \item bounds/intersections are *not* allowed in annotations 
  \begin{lstlisting}
foo (
  add : (Int -> Int) & (Str -> Str)
) : T = ...
  \end{lstlisting}
  \item to avoid backtracking search in constraint solving 
  \begin{lstlisting}
(Int -> Int) & (Str -> Str) <: U
  \end{lstlisting}
  \end{itemize}
\end{frame}

\begin{frame}[fragile]
  \frametitle{Leaky Bound Variable}


  \begin{itemize}
  \item recall 
  \begin{lstlisting}
foo f = (f 123, f True)
  \end{lstlisting}
  \item consider the expression 
  \begin{lstlisting}
fn x => foo (fn y => x (y, y))
  \end{lstlisting}
  \item the inner function's type is generalized 
  \begin{lstlisting}
(fn y => x (y, y)) : 
ALL b . b -> c 
  \end{lstlisting}
  \item unsound if bound variable leaks into outer constraint 
  \begin{lstlisting}
x : a, a <: (b,b) -> c 
  \end{lstlisting}
  \end{itemize}
\end{frame}

\begin{frame}[fragile]
  \frametitle{Subtype Extrusion}

  \begin{itemize}
  \item extrude types that are too polymorphic
  \begin{lstlisting}
(fn y => x (y, y)) : 
ALL b {b <: b'} . b -> c 
  \end{lstlisting}
  \item constrain outer param with extruded type  
  \begin{lstlisting}
a <: (b',b') -> c
  \end{lstlisting}
  \item generate instantiated constraints 
  \begin{lstlisting}
a <: (b',b') -> c, Int <: b', Bool <: b' 
  \end{lstlisting}
  \item or as union 
  \begin{lstlisting}
a <: (Int|Bool, Int|Bool) -> c
  \end{lstlisting}
  \end{itemize}
\end{frame}

\begin{frame}[fragile]
  \frametitle{Transitive Closure}

  \begin{itemize}
  \item suppose some constraint has already been found 
  \begin{lstlisting}
L <: a
  \end{lstlisting}
  \item and a new constraint is discovered 
  \begin{lstlisting}
a <: U 
  \end{lstlisting}
  \item then solve transitive constraint  
  \begin{lstlisting}
L <: U
  \end{lstlisting}
  \end{itemize}
\end{frame}

\begin{frame}[fragile]
  \frametitle{Instantiating Left Parametric Type}

  \begin{itemize}
  \item solve constraint with parametric type on the left 
  \begin{lstlisting}
ALL a {L <: U} . T <: V
  \end{lstlisting}
  \item free the variables and solve apparent constraints 
  \begin{lstlisting}
[a := a']L <: [a := a']U, [a := a']T <: V 
  \end{lstlisting}
  \end{itemize}
\end{frame}


\begin{frame}[fragile]
  \frametitle{Freezing Right Parametric Type}

  \begin{itemize}
  \item solve constraint with parametric type on the right 
  \begin{lstlisting}
a -> b <: ALL c . c -> c  
  \end{lstlisting}
  \item treat the variable as "skolem" 
  \begin{lstlisting}
c_sk <: a, b <: c_sk  
  \end{lstlisting}
  \item interpret skolem conservatively 
  \begin{lstlisting}
TOP <: a, b <: BOT  
  \end{lstlisting}
  \end{itemize}
\end{frame}

\begin{frame}[fragile]
  \frametitle{Main Ideas}

  \begin{itemize}
  \item use intersection and union to infer satisfactory types 
  \item limit intersection/union to negative/positive positions 
  \item solve for variable bounds in subtyping constraints  
  \end{itemize}
\end{frame}

\end{document}